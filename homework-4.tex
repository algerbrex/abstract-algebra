\documentclass[12pt, letterpaper]{article}
\usepackage{amsfonts}
\usepackage{amssymb}
\usepackage{amsmath}
\usepackage{amsthm}
\usepackage{titling}
\usepackage{mathtools}
\usepackage[shortlabels]{enumitem}
\usepackage[margin=1in]{geometry}

\setlength{\droptitle}{-8ex}
\pretitle{\begin{flushleft}\large}
\posttitle{\par\end{flushleft}}
\preauthor{\begin{flushleft}\large}
\postauthor{\end{flushleft}}
\predate{\begin{flushleft}\large}
\postdate{\end{flushleft}}

\title{Abstract Algebra (MATH-4620) HOMEWORK 4}
\author{Christian Dean}
\date{October 16, 2023}


\newenvironment{problem}
    [1]
    {\noindent \textbf{Problem #1:}}
    {\vspace{3mm}}

\begin{document}

\maketitle

\noindent\hfil\rule{16cm}{0.4pt}\hfil

\begin{problem}{1}
    \emph{Proof:} Let $H$ be the set of elements from $G$ that have finite order.

    \bigskip\noindent
    $H$ contains the identity: Note that $1$ is the smallest natural number such that 
    $e^1 = e$. Thus the order of $e$ is $1$, and thus finite. So $e \in H$.

    \bigskip\noindent
    $H$ is also closed under the group operation: Let $h, g \in H$. Then let $|{<}h{>}| = n,
    |{<}g{>}| = m$ for some $n, m \in \mathbb{N}$. Let $k = nm$. Then: $(hg)^k = (h^k)(g^k) = 
    (h^{nm})(g^{nm}) = ((h^n)^m)((g^m)^n) = (e^m)(e^n) = (e)(e) = e$. Thus as $(hg)^k = e$, then
    $|{<}hg{>}| \le k$. Since, if there exists a smaller natural number $l < k$, such
    that $(hg)^l = e$, then by \textbf{Theorem 8}, $|{<}hg{>}| = l < k$. And if no smaller 
    natural number $l < k$ is such that $(hg)^l = e$, then again by \textbf{Theorem 8}, 
    $|{<}hg{>}| = k$. Thus the order of $hg$ is finite.

    \bigskip\noindent
    $H$ is also closed under inverses: Let $h \in H$. Then $h^n = e$ for some $n \in \mathbb{N}$.
    Then:
        \begin{align*}
            &h^n = e\\
            \Rightarrow \; &h^{-n}h^n = h^{-n}e\\
            \Rightarrow \; &e = h^{-n}\\
        \end{align*}
    And by similar reasoning to the above proof that $H$ is closed under the group operation,
    $|{<}h^{-1}{>}| \leq n$ and thus $h^{-1}$, has finite order. (My guess is that in fact 
    $|{<}h^{-1}{>}| = n$, but I wasn't sure if that was implied by the above proof and didn't
    want to make too strong of a claim).
\end{problem}

\begin{problem}{2}
    \begin{enumerate}
        \setcounter{enumi}{1}

        \item $\phi$ is not a homomorphism. Counterexample: $\phi(2.9 + 5.7) = \lfloor8.6\rfloor =
        8$ but $\phi(2.9) + \phi(5.7) = \lfloor2.9\rfloor + \lfloor5.7\rfloor = 2 + 5 = 7$.

        \item $\phi$ is a homomorphism: Let $a, b \in \mathbb{R}^*$. Then $\phi(ab) = |ab| = 
        |a||b| = \phi(a)\phi(b)$.

        \item $\phi$ is a homomorphism: Let $a, b \in \mathbb{Z}_6$. \textbf{Case 1}:
        $a$ and $b$ have the same parity. Then $a + b \; (\text{mod } 6)$ is even, and thus 
        $\phi(a + b \; (\text{mod } 6)) = 0$. If $a$ and $b$ are both odd, then $\phi(a) 
        + \phi(b) = 1 + 1 \; (\text{mod } 2) = 0$. And if $a$ and $b$ are both even then 
        $\phi(a) + \phi(b) = 0 + 0 \; (\text{mod } 2) = 0$. \textbf{Case 2:} Without loss
        of generality suppose $a$ is odd and $b$ is even. Then $a + b \; (\text{mod } 6)$
        is odd, and $\phi(a + b \; (\text{mod } 6)) = 1$. And $\phi(a) + \phi(b) = 1 + 0
        \; (\text{mod } 2) = 1$. Thus in both cases, $\phi(a + b \; (\text{mod } 6)) = \phi(a) 
        + \phi(b)$.

        \item $\phi$ is not a homomorphism as $|{<}\phi(1){>}| = 2$ and thus
        does not divide $|{<}1{>}| = 9$.

        \setcounter{enumi}{8}

        \item $\phi$ is a homomorphism as the differention operation is linear.
        
        \setcounter{enumi}{11}

        \item $\phi$ is not a homomorphism: Let $A = \begin{bmatrix*}[r] 1 & 2 \\ 3 & 4 
        \end{bmatrix*}$ and $A = \begin{bmatrix*}[r] 5 & 6 \\ 7 & 8 \end{bmatrix*}$. Then
        $\phi(A + b) = \phi(\begin{bmatrix*}[r] 6 & 9 \\ 9 & 12 \end{bmatrix*}) = -9$.
        But $\phi(A) + \phi(B) = -2 + (-2) = -4$.

        \item $\phi$ is a homomorphism: Let $A, B \in M_2$. Then $\phi(A + B) = 
        (a_{11} + b_{11}) + (a_{22} + b_{22}) = (a_{11} + a_{22}) + (b_{11} + b_{22}) = 
        \phi(A) + \phi(B)$.
    \end{enumerate}
\end{problem}

\begin{problem}{3}
    \begin{enumerate}
        \setcounter{enumi}{2}

        \item $\text{Ker}(\phi) = \{1, -1\}$

        \item $\text{Ker}(\phi) = \{0, 2, 4\}$

        \setcounter{enumi}{8}

        \item $\text{Ker}(\phi) = \{f \in F \;|\; f^{''}(x) = 0 \}$
        
        \setcounter{enumi}{12}

        \item $\text{Ker}(\phi) = \{A \in M_2 \;|\; a_{11} = -a_{22}\}$
    \end{enumerate}
\end{problem}

\begin{problem}{4}
    \begin{enumerate}
        \setcounter{enumi}{16}

        \item Let $x \in \mathbb{Z}$. Then $\phi(x) = 4x \;(\text{mod } 7)$. Because, $x$ can be 
        partioned into $x$ $1$'s. So $\phi(x) = \phi(1 + 1 + \ldots + 1) = \phi(1) + \phi(1) + 
        \ldots + \phi(1) = 4 + 4 + \ldots + 4 \;(\text{mod } 7) = 4x \;(\text{mod } 7)$. Thus,
        $\text{Ker}(\phi)$ will contain all integers $x$ such that $7 | 4x$. And as $4$ has no 
        factor of $7$, $\text{Ker}(\phi)$ must be all integers $x$ that contain a factor $7$,
        which implies $\text{Ker}(\phi) = 7\mathbb{Z}$.

        \bigskip\noindent
        And $\phi(25) = 4*25 \; (\text{mod } 7) = 2$.

        \item By similar reasoning in the answer to (17), $\phi(x) = 6x \;(\text{mod } 10)$ for any $x \in \mathbb{Z}$.
        Thus, $\text{Ker}(\phi)$ will contain all integers $x$ such that $10 | 6x$. And as the prime
        factorization of $10 = 2 * 5$ and $6 = 2 * 3$, $\text{Ker}(\phi)$ will contain all integers 
        that have a factor of $5$. This implies $\text{Ker}(\phi) = 5\mathbb{Z}$.

        \bigskip\noindent
        And $\phi(18) = 6*18 \; (\text{mod } 10) = 8$.

        \setcounter{enumi}{22}

        \item Let $(x, y) \in \mathbb{Z} \times \mathbb{Z}$. Then $\phi((x, y)) = \phi((x, 0) + (0, y)) =
        \phi((x, 0)) + \phi((0, y)) = \phi((1, 0) + \ldots + (1, 0)) + \phi((0, 1) + \ldots + (0, 1)) = 
        \phi((1, 0)) + \ldots + \phi((1, 0)) + \phi((0, 1)) + \ldots + \phi((0, 1)) = (2, -3) + \ldots 
        (2, -3) + (-1, 5) + \ldots + (-1, 5) = (2x, -3x) + (-y, 5y)$.

        \bigskip\noindent
        So, $\text{Ker}(\phi)$ will be all points $(x, y)$ such that $(2x, -3x) + (-y, 5y) = (0, 0)$.
        Which implies it will be all points $(x, y)$ such that:
            \begin{align*}
                2x - y &= 0\\
                5y - 3x &= 0\\
            \end{align*}
        And the only solution for this system of equations is $x = 0, y = 0$. Thus $\text{Ker}(\phi) = 
        \{(0, 0)\}$.

        \bigskip\noindent
        And $\phi((4, 6)) = (2*4, -3*4) + (-6, 5*6) = (8, -12) + (-6, 30) = (2, 18)$.

        \item Let $(x, y) \in \mathbb{Z} \times \mathbb{Z}$. And $\phi((x, y)) = ((3,5)(2,4))^x((1,7)(6,10,8,9))^y$
        by similar reasoning to the solution in (23). Thus $\text{Ker}(\phi)$ will be all points $(x,y)$ where
        $((3,5)(2,4))^x\\((1,7)(6,10,8,9))^y = \text{id}$. And as $|{<}(3,5)(2,4){>}| = \text{lcm}(2, 2) = 2$ and 
        $|{<}(1,7)\\(6,10,8,9){>}| = \text{lcm}(2, 4) = 4$, $\text{Ker}(\phi)$ will be all points $(x, y)$ where $x$
        is a multiple of $2$ and $y$ is a multiple of $4$. Thus, $\text{Ker}(\phi) = 2\mathbb{Z} \times 
        4\mathbb{Z}$.

        \bigskip\noindent
        And: 
        \begin{align*} 
            &\phi((3,10))\\ 
            &= ((3,5)(2,4))^3((1,7)(6,10,8,9))^{10}\\ 
            &= ((3,5)(2,4))^2((3,5)(2,4))^1 ((1,7)(6,10,8,9))^8((1,7)(6,10,8,9))^2\\ 
            &= (\text{id})(3,5)(2,4)(\text{id})((1,7)(6,10,8,9))^2\\ 
            &= (3,5)(2,4)((1,7)(6,10,8,9))^2\\
            &= (2,4)(3,5)(6,8)(9,10)
        \end{align*}
    \end{enumerate}
\end{problem}

\begin{problem}{5}
    \begin{enumerate}
        \setcounter{enumi}{32}
        \item There is no non-trival homomorphism. As ${<}1{>} = \mathbb{Z}_{12}$,
        a homomorphism $\phi: \mathbb{Z}_{12} \rightarrow \mathbb{Z}_5$ is completely
        determined by where it maps $1 \in \mathbb{Z}_{12}$. And by \textbf{Theorem 40}, 
        $\phi(1) | |{<}1{>} = 12$. Thus, $\phi(1)$ can only map to $0 \in \mathbb{Z}_5$,
        and by the homomorphism property, all elements of $\mathbb{Z}_{12}$ will be mapped
        to $0$.

        \item $\phi: \mathbb{Z}_{12} \rightarrow \mathbb{Z}_4$ defined by $\phi(1) = 2$
        is a non-trivial homomorphism.

        \item $\mathbb{Z}_2 \times \mathbb{Z}_4$ is generated by $X = \{(0,1), (1,0)\}$, so 
        a homomorphism will be completely determined by where it maps the elements of $X$.
        $|{<}(0,1){>}| = 4$ and $|{<}(1,0){>}| = 2$, so the order of $\phi((0,1))$
        must divide $4$ and the order of $\phi((1,0))$ must divide $2$.

        \bigskip\noindent
        By the Chinese Remainder Theorem, $\mathbb{Z}_2 \times \mathbb{Z}_5 \cong \mathbb{Z}_{10}$
        by the isomorphism $\phi(x \;(\text{mod 10}))\\ = (x \;(\text{mod 2}), x \;(\text{mod 5}))$.
        Thus $\mathbb{Z}_2 \times \mathbb{Z}_5$ is finite, cyclic and will have elements of order
        $10/1 = 10$, $10/2 = 5$, $10/5 = 2$, and $10/10 = 1$. And as $5$ is the only element in 
        $\mathbb{Z}_{10}$ that has order $2$, then by the isomorphism $\phi(5) = (1, 0)$ is the
        only element in $\mathbb{Z}_2 \times \mathbb{Z}_5$ that has order $2$. Thus, as no other
        element in $\mathbb{Z}_2 \times \mathbb{Z}_5$ will have an order that divides $2$ and $4$,
        then both $(0, 1)$ and $(1, 0)$ must map to $(1, 0) \in \mathbb{Z}_2 \times \mathbb{Z}_5$.

        \bigskip\noindent
        So $\varphi: \mathbb{Z}_2 \times \mathbb{Z}_4 \rightarrow \mathbb{Z}_2 \times \mathbb{Z}_5$
        defined by $\varphi((0, 1)) = \varphi((1, 0)) = (1, 0)$ is a non-trivial homomorphism.

        \item There is no non-trivial homomorphism. As ${<}1{>} = \mathbb{Z}_{3}$,
        a homomorphism $\phi: \mathbb{Z}_{3} \rightarrow \mathbb{Z}$ is completely
        determined by where it maps $1 \in \mathbb{Z}_{3}$. Suppose $\phi(1) = z \in \mathbb{Z} 
        \backslash \{0\}$. Then $\phi(n) = nz$ for any $n \in \mathbb{Z}_3$. Thus $\phi(2) = 2z$. 
        But by \textbf{Theorem 37}, as $2$ is the inverse of $1$ in $\mathbb{Z}_3$, then $\phi(2)$ 
        must map to the inverse of $\phi(1)$, which is $-z$. Thus $\phi(2) = 2z = -z \Rightarrow
        3z = 0 \Rightarrow z = 0$. Thus $\phi(1)$ must map to $0$, and every element
        of $\mathbb{Z}_3$ maps to $0$.

        \item $\phi: \mathbb{Z}_3 \rightarrow S_3$ defined by $\phi(1) = (1,2,3)$ is a non-trivial
        homomorphism.

        \item $\phi: \mathbb{Z} \rightarrow S_3$ defined by $\phi(1) = (1,2)$
        is a non-trivial homomorphism.
    \end{enumerate}
\end{problem}

\begin{problem}{6}
    \begin{enumerate}[(a)]
        \item Prime power decomposition (PPD): 
        
        \begin{enumerate}[1.]
            \item $\mathbb{Z}_3 \times \mathbb{Z}_8$
            \item $\mathbb{Z}_3 \times \mathbb{Z}_2 \times \mathbb{Z}_4$
            \item $\mathbb{Z}_3 \times \mathbb{Z}_2 \times \mathbb{Z}_2 \times \mathbb{Z}_2$
        \end{enumerate}
        
        Increasing divisibility decomposition (IDD):
        \begin{enumerate}[1.]
            \item $\mathbb{Z}_{24}$
            \item $\mathbb{Z}_2 \times \mathbb{Z}_{12}$
            \item $\mathbb{Z}_2 \times \mathbb{Z}_2 \times \mathbb{Z}_{6}$
        \end{enumerate}

        \item PPD: 
        \begin{enumerate}[1.]
            \item $\mathbb{Z}_5 \times \mathbb{Z}_3 \times \mathbb{Z}_4$
            \item $\mathbb{Z}_5 \times \mathbb{Z}_3 \times \mathbb{Z}_2 \times \mathbb{Z}_2$
        \end{enumerate}
        
        IDD:
        \begin{enumerate}[1.]
            \item $\mathbb{Z}_{60}$
            \item $\mathbb{Z}_2 \times \mathbb{Z}_{30}$
        \end{enumerate}

        \item PPD:
        \begin{enumerate}[1.]
            \item $\mathbb{Z}_5 \times \mathbb{Z}_5 \times \mathbb{Z}_5$
            \item $\mathbb{Z}_5 \times \mathbb{Z}_{25}$
            \item $\mathbb{Z}_{125}$
        \end{enumerate}
        
        IDD:
        \begin{enumerate}[1.]
            \item $\mathbb{Z}_5 \times \mathbb{Z}_5 \times \mathbb{Z}_5$
            \item $\mathbb{Z}_5 \times \mathbb{Z}_{25}$
            \item $\mathbb{Z}_{125}$
        \end{enumerate}

        \item PPD:
        \begin{enumerate}[1.]
            \item $\mathbb{Z}_7 \times \mathbb{Z}_7 \times \mathbb{Z}_2 \times \mathbb{Z}_2 
                \times \mathbb{Z}_2 \times \mathbb{Z}_2 \times \mathbb{Z}_2$,
            \item $\mathbb{Z}_7 \times \mathbb{Z}_7 \times \mathbb{Z}_4 \times \mathbb{Z}_2 
                \times \mathbb{Z}_2 \times \mathbb{Z}_2$,
            \item $\mathbb{Z}_7 \times \mathbb{Z}_7 \times \mathbb{Z}_8 \times \mathbb{Z}_2 
                \times \mathbb{Z}_2$,
            \item $\mathbb{Z}_7 \times \mathbb{Z}_7 \times \mathbb{Z}_2 \times \mathbb{Z}_4 \times \mathbb{Z}_4$
            \item $\mathbb{Z}_7 \times \mathbb{Z}_7 \times \mathbb{Z}_4 \times \mathbb{Z}_8$
            \item $\mathbb{Z}_7 \times \mathbb{Z}_7 \times \mathbb{Z}_{16} \times \mathbb{Z}_2$,
            \item $\mathbb{Z}_7 \times \mathbb{Z}_7 \times \mathbb{Z}_{32}$
            
            \item $\mathbb{Z}_{49} \times \mathbb{Z}_2 \times \mathbb{Z}_2 
                \times \mathbb{Z}_2 \times \mathbb{Z}_2 \times \mathbb{Z}_2$,
            \item $\mathbb{Z}_{49} \times \mathbb{Z}_4 \times \mathbb{Z}_2 
                \times \mathbb{Z}_2 \times \mathbb{Z}_2$,
            \item $\mathbb{Z}_{49} \times \mathbb{Z}_8 \times \mathbb{Z}_2 
                \times \mathbb{Z}_2$,
            \item $\mathbb{Z}_{49} \times \mathbb{Z}_2 \times \mathbb{Z}_4 \times \mathbb{Z}_4$
            \item $\mathbb{Z}_{49} \times \mathbb{Z}_4 \times \mathbb{Z}_8$
            \item $\mathbb{Z}_{49} \times \mathbb{Z}_{16} \times \mathbb{Z}_2$,
            \item $\mathbb{Z}_{49} \times \mathbb{Z}_{32}$
        \end{enumerate}
        
        IDD:
        \begin{enumerate}[1.]
            \item $\mathbb{Z}_2 \times \mathbb{Z}_2 \times \mathbb{Z}_2 \times \mathbb{Z}_{14}
                \times \mathbb{Z}_{14}$
            \item $\mathbb{Z}_2 \times \mathbb{Z}_2 \times \mathbb{Z}_{14} \times \mathbb{Z}_{28}$
            \item $\mathbb{Z}_{2} \times \mathbb{Z}_{14} \times \mathbb{Z}_{56}$
            
            \item $\mathbb{Z}_2 \times \mathbb{Z}_{28} \times \mathbb{Z}_{28}$
            \item $\mathbb{Z}_{28} \times \mathbb{Z}_{56}$

            \item $\mathbb{Z}_{14} \times \mathbb{Z}_{112}$
            \item $\mathbb{Z}_7 \times \mathbb{Z}_{224}$
            
            \item $\mathbb{Z}_2 \times \mathbb{Z}_2 \times \mathbb{Z}_2 \times \mathbb{Z}_2 
                \times \mathbb{Z}_{98}$,
            \item $\mathbb{Z}_2 \times \mathbb{Z}_2 \times \mathbb{Z}_2 \times \mathbb{Z}_{196}$
            \item $\mathbb{Z}_2 \times \mathbb{Z}_2 \times \mathbb{Z}_{392}$
            
            \item $\mathbb{Z}_2 \times \mathbb{Z}_4 \times \mathbb{Z}_{196}$
            \item $\mathbb{Z}_4 \times \mathbb{Z}_{392}$

            \item $\mathbb{Z}_2 \times \mathbb{Z}_{784}$
            \item $\mathbb{Z}_{1568}$
        \end{enumerate}

    \end{enumerate}
\end{problem}

\end{document}