\documentclass[12pt, letterpaper]{article}
\usepackage{amsfonts}
\usepackage{amssymb}
\usepackage{amsmath}
\usepackage{amsthm}
\usepackage{titling}
\usepackage{mathtools}
\usepackage[shortlabels]{enumitem}
\usepackage[margin=1in]{geometry}

\setlength{\droptitle}{-8ex}
\pretitle{\begin{flushleft}\large}
\posttitle{\par\end{flushleft}}
\preauthor{\begin{flushleft}\large}
\postauthor{\end{flushleft}}
\predate{\begin{flushleft}\large}
\postdate{\end{flushleft}}

\title{Abstract Algebra (MATH-4620) HOMEWORK 5}
\author{Christian Dean}
\date{October 27, 2023}


\newenvironment{problem}
    [1]
    {\noindent \textbf{Problem #1:}}
    {\vspace{3mm}}

\begin{document}

\maketitle

\noindent\hfil\rule{16cm}{0.4pt}\hfil

\begin{problem}{1}
    \begin{enumerate}[(a)]
        \item $|\mathbb{Z}_9 \times \mathbb{Z}_{35}| = 9 * 35 = 315$ and $|{<}3{>} \times 
        {<}25{>}| = (9/3) * (35/5) = 3 * 7 = 21$. Thus, the order of the factor group is
        $315/21 = 15$.

        \item $|\mathbb{Z}_9 \times \mathbb{Z}_{35}| = 315$ and $|\{0\} \times 
        {<}11{>}| = 1 * (35/1) = 35$. Thus, the order of the factor group is $315/35 = 
        9$.

        \item $|\mathbb{Z}_{18} \times \mathbb{Z}_{24}| = 18 * 24 = 432$. And $|{<}(3,2){>}|$
        is the smallest $n$ such that $n * (3,2) = (3n, 2n) = (0, 0)$ which implies
        it is the smallest $n$ such that $18|3n$ and $24|2n$. And as $18 = 3^2 * 2$ and
        $24 = 2^3 * 3$, then $n = 2^2 * 3 = 12$. Thus the order of the factor group is
        $432 / 12 = 36$.

        \item $|\mathbb{Z}_{18} \times \mathbb{Z}_{24}| = 432$. And $|{<}(1,1){>}|$
        is the smallest $n$ such that $n * (1,1) = (n, n) = (0, 0)$ which implies
        it is the smallest $n$ such that $18|n$ and $24|n$. Thus $n = lcm(18, 24) = 72$. 
        Thus the order of the factor group is $432 / 72 = 6$.

        \item $|\mathbb{Z}_{19} \times \mathbb{Z}_{24}| = 19 * 24 = 456$. And 
        $|{<}(1,1){>}|$ is the smallest $n$ such that $n * (1,1) = (n, n) = (0, 0)$ 
        which implies it is the smallest $n$ such that $19|n$ and $24|n$. Thus 
        $n = lcm(19, 24) = 19 * 2^3 * 3 = 19 * 24 = 456$. Thus the order of the factor 
        group is $456 / 456  = 1$.
    \end{enumerate}
\end{problem}

\begin{problem}{2}
    \begin{enumerate}[(a)]
        \item $|{<}\overline{10}{>}|$ is the smallest $n$ such that $(\overline{10})^n = \overline{0}$,
        which implies it is the smallest $n$ such that $10^n = 10n \in {<}8{>}$. Thus 
        $n = 4$ as $10(4) = 16 \;(\text{mod } 24)$. Thus $|{<}\overline{10}{>}| = 4$.

        \item $|{<}\overline{(1,7)}{>}|$ is the smallest $n$ such that $(1,7)^n \in {<}(1,1){>}$.
        And as $|(\mathbb{Z}_6 \times \mathbb{Z}_9) / ({<}(1,1){>})| \\= 54/18 = 3$,
        then $|{<}\overline{(1,7)}{>}|$ must divide $3$, so it is $1$ or $3$. And thus
        $n = 1$ as $(1,7) = (7 \;(\text{mod } 6), 7 \;(\text{mod } 9)) \in {<}(1,1){>}$.

        \item $|{<}\overline{(2,3)}{>}|$ is the smallest $n$ such that $(2,3)^n \in {<}(1,2){>} = 
        \{(0,0), (1, 2), (2, 4), (3, 6)\}$. And as $|(\mathbb{Z}_4 \times \mathbb{Z}_8) / 
        ({<}(1,2){>})| = 32/4 = 8$, then $|{<}\overline{(2,3)}{>}|$ must divide $8$, 
        so it must be $1$, $2$, $4$, or $8$. And thus, checking each value, $n = 8$ as
        $8 * (2,3) = (16 \;(\text{mod } 4), 24 \;\\(\text{mod } 8)) = (0,0) \in {<}(1,2){>}$.

        \item $|{<}\overline{(3,2)}{>}|$ is the smallest $n$ such that $(3,2)^n \in {<}(1,2){>} = 
        \{(0,0), (1, 2), (2, 4), (3, 6)\}$. And as $|(\mathbb{Z}_4 \times \mathbb{Z}_8) / 
        ({<}(1,2){>})| = 32/4 = 8$, then $|{<}\overline{(3,2)}{>}|$ must divide $8$, 
        so it must be $1$, $2$, $4$, or $8$. And thus, checking each value, $n = 2$ as
        $2 * (3,2) = (6 \;(\text{mod } 4), 4 \;(\text{mod } 8))\\ = (2, 4) \in {<}(1,2){>}$.
        
    \end{enumerate}
\end{problem}

\begin{problem}{3}
    \begin{enumerate}[(a)]
        \item $(\mathbb{Z} \times \mathbb{Z}) / (\langle2\rangle \times \langle4\rangle) \cong 
        (\mathbb{Z} / \langle2\rangle) \times (\mathbb{Z} / \langle4\rangle) = (\mathbb{Z} / 
        2\mathbb{Z}) \times (\mathbb{Z} / 4\mathbb{Z}) \cong \mathbb{Z}_2 \times \mathbb{Z}_4$.

        \item 
        
        \item 
    \end{enumerate}
\end{problem}

\begin{problem}{4}
    \begin{enumerate}[(a)]
        \item \emph{Proof}: Suppose $G$ is a group, $H$ is a normal subgroup, and $G/H$ is infinite.
        As $G/H$ is a partition of G, then $|G|$ is the sum of cardinality of each coset. Let $|G| = 
        S = \sum_{i=1}^{\infty} |C_i|$, where $C_i$ is the i-th coset in $G/H$. As $|C_i| \in 
        \mathbb{N}$ for each $i$, then the infinite sum $S$ diverges to infinity. Thus $G$ is infinite.
        
        \item False. \emph{Counterexample}: $\mathbb{Z} / 2\mathbb{Z} \cong \mathbb{Z}_2$, and so
        $|\mathbb{Z} / 2\mathbb{Z}| = 2$. But $\mathbb{Z}$ is infinite.
        
        \item False. \emph{Counterexample}: $S_6/A_6 \cong \mathbb{Z}_2$, and so $S_6/A_6$ is abelian.
        But $S_6$ is not abelian.
        
        \item \emph{Proof}: This is the contrapositive of \textbf{Theorem 47}, part (ii).

        \bigskip\noindent
        Alternatively: \emph{Proof:} Suppose $G/H$ is non-abelian. Then there exists two cosets, $xH$ and $yH$ 
        such that $xyH \neq yxH$. \textbf{Case 1:} Suppose $xy \notin yxH$. Then $xy \neq yx$ for every $h \in H$,
        which implies $xy \neq yx(e_G) \Rightarrow xy \neq yx$. \textbf{Case 2:} Suppose $xy \in yxH$. Then
        $xy = yxh$ for some $h \in H$. But $h$ cannot be $e_G$, since if it were, $xy = yx \Rightarrow
        xyH = yxH$, a contradiction. Thus $h \neq e_G$, and thus $xy \neq yx$. And as there exists two elements in
        $G$, $x$ and $y$ such that they do not commute, the proof is complete.
        
        \item
    \end{enumerate}
\end{problem}

\end{document}