\documentclass[12pt, letterpaper]{article}
\usepackage{amsfonts}
\usepackage{amssymb}
\usepackage{amsmath}
\usepackage{amsthm}
\usepackage{titling}
\usepackage{mathtools}
\usepackage[shortlabels]{enumitem}
\usepackage[margin=1in]{geometry}

\setlength{\droptitle}{-8ex}
\pretitle{\begin{flushleft}\large}
\posttitle{\par\end{flushleft}}
\preauthor{\begin{flushleft}\large}
\postauthor{\end{flushleft}}
\predate{\begin{flushleft}\large}
\postdate{\end{flushleft}}

\title{Abstract Algebra (MATH-4620) HOMEWORK 2}
\author{Christian Dean}
\date{September 1, 2023}


\newenvironment{problem}
    [1]
    {\noindent \textbf{Problem #1.}}
    {\vspace{3mm}}


\newenvironment{solution}
    [0]
    {\noindent \textbf{Solution:}} 
    {\vspace{3mm}}


\begin{document}

\maketitle

\noindent\hfil\rule{16cm}{0.4pt}\hfil

\begin{problem}{1}
    Which of the following groups are cyclic? For each cyclic group, list all the generators of the group.
    \begin{align*}
        &G_1 = (\mathbb{Z}, +) \quad G_2 = (\mathbb{Q}, +) \quad G_3 = (\mathbb{Q}^+, \cdot) \quad G_4 = (6\mathbb{Z}, +)\\
        &G_5 = \{6^n | n \in \mathbb{Z}\} \text{ under multiplication}\\
        &G_6 = \{a+b\sqrt{2}\ | a, b \in \mathbb{Z}\} \text{ under addition}
    \end{align*}
\end{problem}

\begin{solution}
    \begin{enumerate}[(1)]
        \item $G_1$ is a cyclic group whose generator is $1$.
        \item $G_2$ is not cyclic. \emph{Proof:} $\mathbb{Q}$ being dense in $\mathbb{R}$
        implies that between any two rational numbers, another rational number exists.
        Thus, between any two succesive elements, in any cyclic subgroup of $(\mathbb{Q}, +)$,
        a rational number exists that is in $\mathbb{Q}$ but not in said cyclic subgroup. 
        \item $G_3$ is not a cyclic group. A proof similar to the one in (2) holds here.
        \item $G_4$ is a cyclic group whose generator is $6$.
        \item $G_5$ is a cyclic group whose generators are $\frac{1}{6}$ and $6$.
        \item $G_6$ is not a cyclic group. \emph{Proof:} $(G_6, +) \cong \mathbb{Z}^2 = 
        \{\begin{bmatrix} a \\ b \end{bmatrix}\ | a, b \in \mathbb{Z}\}$ equipped with 
        coordinate-wise vector addition. And as cyclicity is preserved through isomorphisms,
        it suffices to prove $\mathbb{Z}^2$ is not cyclic. Proving $\forall \vec{v} \in
        \mathbb{Z}^2, span(\{\vec{v}\}) \neq \mathbb{Z}^2$ is equivalent to proving $\mathbb{Z}^2$
        is not cyclic. This is because a linear combination of a single vector $\vec{x}$, 
        $c\vec{x}$ for some $c \in \mathbb{Z}$, equals $(\vec{x})^c$, where exponenitation 
        denotes repeatd application of the group operator, coordinate-wise vector addition. Thus
        no vector in $\mathbb{Z}^2$ being able to span the vector space implies that $\mathbb{Z}^2$
        can not be generated by a single element, and is thus not cyclic. Using the theorem that
        $n-1$ vectors cannot span an $n$-dimensional vector space, $\forall \vec{v} \in \mathbb{Z}^2, 
        span(\{\vec{v}\}) \neq \mathbb{Z}^2$ is proven. And thus $G_6$ is not cyclic.
    \end{enumerate}
\end{solution}

\begin{problem}{2}
    Find all subgroups of the given groups and draw their subgroup diagrams: 
    $\mathbb{Z}_{12}$, $\mathbb{Z}_{36}$, $\mathbb{Z}_{8}$.
    
\end{problem}

\begin{solution}
    \vspace{250mm}
\end{solution}

\begin{problem}{3}
    Below is a list of groups. List them according to their subgroup relationship.
    \begin{align*}
        &G_1 = (\mathbb{Z}, +)\\
        &G_2 = (5\mathbb{Z}, +)\\
        &G_3 = (20\mathbb{Z}, +)\\
        &G_4 = (\mathbb{R}, +)
    \end{align*}
\end{problem}

\begin{solution}
    The following subgroup relationship holds: $G_3 \leq G_2 \leq G_1 \leq G_4$.
\end{solution}

\begin{problem}{4}
    For each of the following, determine whether the given set is a group under the 
    given operation. Justify your answer.
\end{problem}

\begin{solution}
    \begin{enumerate}[(a)]
        \item $\star$ defined on $3\mathbb{Z}$ by $a \star b = a + b$: Yes. 
        \emph{Proof:} $0$ is the identity element. Normal addition over
        the integers is associative. And any element $a \in 3\mathbb{Z}$, has the
        form $a = 3k$, for some $k \in \mathbb{Z}$. Let $a^{-1} = 3(-k) = -3k$. By defintion,
        $-3k \in 3\mathbb{Z}$, and $aa^{-1} = a + a^{-1} = 3k + (-3k) = 0$.

        \item $\star$ defined on $\mathbb{R}^*$ by $a \star b = a^b$: No. \emph{Proof:}
        consider $2, 3, 4 \in \mathbb{R}^*$. $2 \star (3 \star 4) = 2^{81}$ but
        $(2 \star 3) \star 4 = 2^{12}$. So $\star$ is not associative.

        \item $\star$ defined on $\mathbb{Q}$ by $a \star b = ab$: No. The only candidate
        for an identity element is $1$. But any rational times $0$ equals $0$, so $0$ has
        no inverse.

        \item $\star$ defined on $\mathbb{Q}^*$ by $a \star b = ab/10$: Yes. $10$ is the 
        identity element. Let $a \in \mathbb{Q}^*$. Then $a \star 10 = a10/10 = a = 10a/10 =
        10 \star a$. $\star$ is associative. Let $a, b, c \in \mathbb{Q}^*$. Then, $a \star
        (b \star c) = a \star (bc/10) = (a)(bc/10)/10 = abc/100 = (ab/10)(c)/10 = (ab/10) 
        \star c = (a \star c) \star c$. And every element in $\mathbb{Q}^*$ has an inverse.
        Let $a \in \mathbb{Q}^*$. Then let $a^{-1} = 100/a$. As $a$ is rational and non-zero, 
        $a^{-1}$ is rational and non-zero and thus in $\mathbb{Q}^*$, and $aa^{-1} = 
        (a)(100/a)/10 = 100 / 10 = 10$.

        \item $\star$ defined on $\mathbb{Q}^*$ by $a \star b = |ab|$: Yes. $1$ is the identity
        element. Normal multiplication over the non-zero rationals is associative. And for any
        element $a \in \mathbb{Q}^*$, define $a^{-1} = 1/a$. As $a$ is rational and non-zero,
        $a^{-1}$ is rational and non-zero and thus in $\mathbb{Q}^*$, and $aa^{-1} = 
        |a \cdot 1/a| = |1| = 1$.
    \end{enumerate}
\end{solution}

\begin{problem}{5}
    Let $A = \begin{pmatrix*}[r] 0 & 1 \\ -1 & 0 \end{pmatrix*}$
    \begin{enumerate}[(a)]
        \item Describe all elements of the cyclic subgroup of $GL_2(\mathbb{R})$ generated
        by $A$.
        \item Do the same as part (a), but considering $A$ as an element of the additive
        group $M_2(\mathbb{R})$.
    \end{enumerate}
\end{problem}

\begin{solution}
    \begin{enumerate}[(a)]
        \item 
        $$
        {<}A{>} = \{
            \begin{pmatrix*}[r] 1 & 0 \\ 0 & 1 \end{pmatrix*}, 
            \begin{pmatrix*}[r] 0 & 1 \\ -1 & 0 \end{pmatrix*}, 
            \begin{pmatrix*}[r] -1 & 0 \\ 0 & -1 \end{pmatrix*},
            \begin{pmatrix*}[r] 0 & -1 \\ 1 & 0 \end{pmatrix*}
        \}
        $$ 
        
        \item
        $$
        {<}A{>} = \{
            \begin{pmatrix*}[r] 0 & n \\ -n & 0 \end{pmatrix*} | n \in \mathbb{Z} 
        \}
        $$
    \end{enumerate}
\end{solution}

\begin{problem}{6}
    Let $G = \mathbb{C}^*$, which is a group under multiplication.
    \begin{enumerate}[(a)]
        \item What is the order of $\dfrac{1 + i\sqrt{3}}{2}$.
        \item What is the order of $1 + i\sqrt{3}$.
    \end{enumerate}
\end{problem}

\begin{solution}
    \begin{enumerate}[(a)]
        \item 
        $$
        {<}\dfrac{1 + i\sqrt{3}}{2}{>} = \{
            1,
            \dfrac{1 + i\sqrt{3}}{2},
            \dfrac{-1 + i\sqrt{3}}{2},
            -1,
            \dfrac{-1 - i\sqrt{3}}{2},
            \dfrac{1 - i\sqrt{3}}{2}
        \}
        $$

        \item No obvious pattern:
        $$
        {<}1 + i\sqrt{3}{>} = \{
            (1 + i\sqrt{3})^n | n \in \mathbb{Z}
        \}
        $$
    \end{enumerate}
\end{solution}

\begin{problem}{7}
    Prove $H = \{\sigma \in S_4 | \sigma(3) = 3\} \le S_4$. Is $H$
    isomorphic to a more easily defined group?
\end{problem}

\begin{solution}
    \emph{Proof:} 

    \bigskip\noindent
    $H$ contains the identity element. $\sigma(x)=x \in H$ as $\sigma(3)=3$ by definition. 
    
    \bigskip\noindent
    $H$ is closed under the group operator, function composition. 
    Let $\tau, \sigma \in H$. Thus, $\tau(3)=3$ and $\sigma(3)=3$. Consider 
    $\tau\sigma$. $\tau\sigma(3) = \tau(3) = 3 \Rightarrow \tau\sigma
    \in H$.

    \bigskip\noindent
    $H$ is closed under inverses. Let $\sigma \in H$. As $\sigma$ is a bijection,
    the inverse of $\sigma$, $\sigma^{-1}$ exists. And as $\sigma^{-1}$ is the inverse
    of $\sigma$, $\sigma^{-1}(\sigma(x)) = x$ for any $x$. Thus, $\sigma^{-1}(\sigma(3)) = 
    3 \Rightarrow \sigma^{-1}(3) = 3$. Thus $\sigma^{-1} \in H$.
\end{solution}

\end{document}