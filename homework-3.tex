\documentclass[12pt, letterpaper]{article}
\usepackage{amsfonts}
\usepackage{amssymb}
\usepackage{amsmath}
\usepackage{amsthm}
\usepackage{titling}
\usepackage{mathtools}
\usepackage[shortlabels]{enumitem}
\usepackage[margin=1in]{geometry}

\setlength{\droptitle}{-8ex}
\pretitle{\begin{flushleft}\large}
\posttitle{\par\end{flushleft}}
\preauthor{\begin{flushleft}\large}
\postauthor{\end{flushleft}}
\predate{\begin{flushleft}\large}
\postdate{\end{flushleft}}

\title{Abstract Algebra (MATH-4620) HOMEWORK 3}
\author{Christian Dean}
\date{September 1, 2023}


\newenvironment{problem}
    [1]
    {\noindent \textbf{Problem #1.}}
    {\vspace{3mm}}


\newenvironment{solution}
    [0]
    {\noindent \textbf{Solution:}} 
    {\vspace{3mm}}


\begin{document}

\maketitle

\noindent\hfil\rule{16cm}{0.4pt}\hfil

\begin{problem}{1}
    Partition the following collection of groups into subcollections of isomorphic groups.
    $$
    \begin{matrix*}[l]
        \mathbb{Z} \text{ under addition} & S_2\\
        
        \mathbb{Z}_6 & \mathbb{R}^* \text{ under multiplication}\\
        
        \mathbb{Z}_2 & \mathbb{R}^+ \text{ under multiplication}\\
        
        S_6 & \mathbb{Q}^* \text{ under multiplication}\\
        
        17\mathbb{Z} \text { under addition} & \mathbb{C}^* \text{ under multiplication}\\
        
        \mathbb{Q} \text{ under addition} & \text{ The subgroup } {<}\pi{>}
        \text{ of }\mathbb{R}^* \text{ under multiplication}\\

        3\mathbb{Z} \text{ under addition} & \text{The subgroup } G \text{ of } S_5
        \text{ generated by } \begin{pmatrix*}[r] 
            1 & 2 & 3 & 4 & 5\\
            3 & 5 & 4 & 1 & 2
        \end{pmatrix*}\\

        \mathbb{R} \text{ under addition}
    \end{matrix*}
    $$
\end{problem}

\begin{solution}
    \begin{itemize}
        \item $\mathbb{Z} \cong 3\mathbb{Z} \cong 17\mathbb{Z} \cong {<}\pi{>}$ as all of the groups are 
        cyclic and have infinite order, so they must all be isomorphic to $\mathbb{Z}$.
        
        \item $\mathbb{Z}_6 \cong G$ as $G$ is generated by $\begin{pmatrix*}[r] 1 & 2 & 3 & 4 & 5\\
        3 & 5 & 4 & 1 & 2 \end{pmatrix*} = (1, 3, 4)(2, 5) \Rightarrow |G| = lcm(2, 3) = 6$. Thus
        as $G$ is a cyclic group of order 6 it must be isomorphic to $\mathbb{Z}_6$.
        
        \item $\mathbb{Z}_2 \cong S_2$ as $S_2$ is a cyclic group, generated by $\sigma = (1, 2) \in S_2$, 
        and has order 2, so it must be isomorphic to $\mathbb{Z}_2$.

        \item $S_6$ is isomorphic to no other group as it has order $6! = 720$, and no other
        group has an order of $720$. Thus no possible bijection can be defined.

        \item $\mathbb{Q}$ and $\mathbb{Q}*$ can only possibly be isomorphic to each other 
        as they are the only non-cylic groups with countably infinite sets. But they cannot be 
        isomorphic as $\mathbb{Q}^*$ has a cyclic subgroup ${<}{-}1{>} = \{1, -1\}$ of order 2,
        but $\mathbb{Q}$ has no such cyclic subgroup.

        \item $\mathbb{C}^*$ could only be isomorphic to the other groups with uncountably infinite
        sets, $\mathbb{R}^*$, $\mathbb{R}^+$, or $\mathbb{R}$. But it is isomorphic to none of them as
        $\mathbb{C}^*$ has a cyclic subgroup of order 4, ${<}i{>} = \{1, i, -1, -i, 1\}$, but none of 
        the other groups with uncountably infinite sets have such a subgroup.

        \item $\mathbb{R}^*$ could only be isomorphic to $\mathbb{R}^+$ and $\mathbb{R}$ as they
        are the only other groups with uncountably infinite sets. But it is not as it also has a 
        subgroup ${<}{-}1{>}$ that has an order of 2, but $\mathbb{R}^+$ and $\mathbb{R}$ do not.
        
        \item Wasn't sure if $\mathbb{R} \cong \mathbb{R}^+?$
    \end{itemize}
\end{solution}

\begin{problem}{2}
    \begin{enumerate}
        \setcounter{enumi}{29}

        \item Let $\sigma$ be a permutation of a set $A$. We shall say ``$\sigma$ moves $a \in A$'' if 
        $\sigma(a) \neq a$. If $A$ is a finite set, how many elements are moved by a cycle $\sigma \in 
        S_A$ of length $n$.

        \item Let $A$ be an infinite set. Let $H = \{\sigma \in A \;|\; \sigma \text{ moves a finite number of elements}\}$.
        Prove that $H \le S_A$.

        \item Let $A$ be an infinite set. Let $K = \{\sigma \in A \;|\; \sigma \text{ moves at most 50 elements}\}$.
        Is $K \le S_A$?
    \end{enumerate}
\end{problem}

\begin{solution}
    \begin{enumerate}
        \setcounter{enumi}{29}

        \item If $n = 1$, then no elements are moved as the single element is mapped back to itself.
        If $n \ge 2$, then $n$-elements are moved by $\sigma$, as each of the $n$ elements are mapped
        to one of the other $n$ elements, besides itself.

        \item \emph{Proof:} $H$ contains the identity permutation as it moves 0 elements, a finite number. 
        
        \bigskip\noindent
        $H$ is also  closed under the group operation, function composition: Let $\sigma, \tau \in H$. Let $\sigma = 
        C_1 C_2 \ldots C_n$ and $\tau = K_1 K_2 \ldots K_m$, where $C_1 C_2 \ldots C_n$ are pairwise disjoint, and 
        $K_1 K_2 \ldots K_m$ are pairwise disjoint. The number of elements $\sigma$ moves will be the sum of its 
        cycle sizes, call it $N$. And the number of elements $\tau$ moves will be the sum of its cycle sizes, call 
        it $M$.
        
        \bigskip\noindent
        If all the cycles, $C_1, K_1, K_2, C_2, \ldots, C_{n-1}, K_{n-1}, C_n, K_n$, are pairwise disjoint, 
        then the number of elements $\tau\sigma$ will move is $N + M$. This is because $\sigma$ will move
        $N$ elements, and $\tau$ won't remap any of those $N$ elements. And $\sigma$ won't remap any of the 
        $M$ elements that $\tau$ moves. Thus $N + M$ total elements will be moved. And if some of the cycles,
        $C_1, K_1, K_2, C_2, \ldots, C_{n-1}, K_{n-1}, C_n, K_n$, are not disjoint, either all of the elements
        in the cycles will still be moved, or some elements might be re-mapped to themselves, in which case less
        than $N + M$ elements will be moved. But in either case, a finite number of elements will be moved.

        \bigskip\noindent
        $H$ is also closed under inverses: Let $\sigma \in H$. Let $M = \{\sigma(a) \;|\; a \in A \text{ and } \sigma(a) \neq a\}$.
        As $\sigma \in H$, $|M|$ is finite, call it $N$. And note the inverse of $\sigma$ will take each element of $M$ and map 
        it back to its preimage, which is distinct from the element itself. Thus the inverse of $\sigma$ will move $N$ elements,
        a finite number.
        
        \item No as it is not closed under function composition. Let $\sigma = (a_2, a_4, a_6, \ldots, a_{100})$ and
        $\tau = (a_1, a_3, a_5, \ldots, a_{99})$, where $a_n \in A, \forall n \in \mathbb{N}$. Note $\sigma, \tau \in K$,
        as each permutation is a cycle that moves $50$ elements. And as $\sigma$ and $\tau$ are disjoint cycles, the number of 
        elements moved by $\tau\sigma$ will be the size of each cycle. This is because each cycle will only move its own elements,
        and leave the elements of the other cycle unaffected. Thus the number of elements moved by $\tau\sigma$ is 
        $100$, and thus $\tau\sigma \notin K$.
    \end{enumerate}
\end{solution}

\begin{problem}{3}
    Let $H = \{\sigma \in S_5 \;|\; \sigma(3) = 4\}$. Is $H \le S_5$?
\end{problem}

\begin{solution}
    No. \emph{Proof:} $H$ fails all conditions for being a subgroup. It does not contain the identity
    permutation as the identity permutation by definition does not map $3$ to $4$. It is also not closed
    under function composition. Let $\sigma = (3, 4)$. Then $\sigma^2$ must map $3$ to $4$. But
    $\sigma^2(3) = \sigma(4) = 3$. It is also not closed under inverses. Consider $\tau = (2, 3, 4, 5)$.
    It's inverse is $\tau^{-1} = (2, 5, 4, 3)$, but $\tau^{-1} \notin H$ as $\tau^{-1}(3) = 2$.
\end{solution}

\begin{problem}{4}
    Write the following permutations as a product of disjoint cycles.
    \begin{enumerate}[(a)]
        \item $\begin{pmatrix*}[r] 1 & 2 & 3 & 4 & 5 \\ 5 & 4 & 1 & 2 & 3 \end{pmatrix*}$
        \item $\begin{pmatrix*}[r] 1 & 2 & 3 & 4 & 5 & 6 & 7 \\ 2 & 4 & 1 & 7 & 3 & 6 & 5 \end{pmatrix*}$
        \item $\begin{pmatrix*}[r] 1 & 2 & 3 & 4 & 5 & 6 & 7 & 8 & 9 & 10 \\ 1 & 4 & 6 & 2 & 8 & 7 & 3 & 9 & 10 & 5 \end{pmatrix*}$
    \end{enumerate}
\end{problem}

\begin{solution}
    \begin{enumerate}[(a)]
        \item (1, 5, 3)(2, 4)
        \item (1, 2, 4, 7, 5, 3)
        \item (2, 4)(3, 6, 7)(5, 8, 9, 10)
    \end{enumerate}
\end{solution}

\begin{problem}{5}
    Write each permutation using "two-rows" notation.
    \begin{enumerate}[(a)]
        \item $(1, 5)(2, 3, 4) \in S_5$
        \item $(1, 4)(2, 4)(2, 3) \in S_6$
        \item $(1, 2, 4, 5)(2, 5, 3)(4, 6, 7) \in S_7$
    \end{enumerate}
\end{problem}

\begin{solution}
    \begin{enumerate}[(a)]
        \item $\begin{pmatrix*}[r] 1 & 2 & 3 & 4 & 5 \\ 5 & 3 & 4 & 2 & 1 \end{pmatrix*}$
        \item $\begin{pmatrix*}[r] 1 & 2 & 3 & 4 & 5 & 6 \\ 4 & 3 & 1 & 2 & 5 & 6 \end{pmatrix*}$
        \item $\begin{pmatrix*}[r] 1 & 2 & 3 & 4 & 5 & 6 & 7 \\ 2 & 1 & 4 & 6 & 3 & 7 & 5 \end{pmatrix*}$
    \end{enumerate}
\end{solution}

\begin{problem}{6}
    Let $\sigma$ and $\tau$ be the following elements of $S_6$:

    $$
    \sigma = \begin{pmatrix*}[r] 1 & 2 & 3 & 4 & 5 & 6 \\ 3 & 1 & 4 & 5 & 6 & 2 \end{pmatrix*} \text{ and }
    \tau = \begin{pmatrix*}[r] 1 & 2 & 3 & 4 & 5 & 6 \\ 2 & 4 & 1 & 3 & 6 & 5 \end{pmatrix*}
    $$

    \noindent
    Find each of the following:

    \begin{enumerate}[(a)]
        \item The order of $\sigma$
        \item $\sigma^{-1}$
        \item $\sigma\tau$
        \item $\tau\sigma$
        \item $\sigma^{-2}\tau^3$
        \item $\sigma^{1000}$
        \item The order of $\tau^{2}$
    \end{enumerate}
\end{problem}

\begin{solution}
    \begin{enumerate}[(a)]
        \item $\sigma = (1, 3, 4, 5, 6, 2) \Rightarrow |\sigma| = lcm(6) = 6$
        \item $\sigma^{-1} = (1, 2, 6, 5, 4, 3)$
        \item $\sigma\tau = (2, 5)$
        \item $\tau\sigma = (4, 6)$
        \item $\sigma^{-2} = (1, 6, 4)(2, 5, 3)$ and $\tau^3 = (1, 3, 4, 2)(5, 6)$
        $\Rightarrow \sigma^{-2}\tau^3 = (1, 2, 6, 3)(4, 5)$
        \item $\sigma^{1000} = \sigma^{996}\sigma^4 = \sigma^{6(166)}\sigma^4 = \sigma_{\text{id}}\sigma^4 = \sigma^4$
        \item $\tau^2 = (1, 4)(2, 3)$
    \end{enumerate}
\end{solution}

\begin{problem}{7}
    Let $G$ be a group and $H$ and $K$ be subgroups of $G$.
    \begin{enumerate}[(a)]
        \item Prove that $H \cap K \le G$.
        \item Must $H \cup K \le G$? If so, prove it; If not,
        find a counterexample.
    \end{enumerate}
\end{problem}

\begin{solution}
    \begin{enumerate}[(a)]
        \item \emph{Proof:} Let $G$ be a group and $H$ and $K$ be subgroups of $G$.
        
        \bigskip\noindent
        As $H$ and $K$ are subgroups, they both contain the identity. 
        So the identity will be in their intersection. 

        \bigskip\noindent
        Let $a, b \in H \cap K$. This implies $a, b \in H$ and $a, b \in K$. And
        as $H$ and $K$ are both subgroups, they are closed under the operation of $G$.
        Thus $ab \in H$ and $ab \in K \Rightarrow ab \in H \cap K$.

        \bigskip\noindent
        Let $a \in H \cap K$. This implies $a \in H$ and $a \in K$. And
        as $H$ and $K$ are both subgroups, they are closed under inverses.
        Thus $\exists \; a^{-1}_H \in H \ni aa^{-1}_H = a^{-1}_Ha = e$ and $\exists \; a^{-1}_K \in 
        K \ni aa^{-1}_K = a^{-1}_Ka = e$. And as each elements inverse is unique,  $a^{-1}_H = a^{-1}_G$,
        and thus the inverse is in both $H$ and $K$. So the inverse is in $H \cap K$.

        \item No. Let $G = (\mathbb{Z}, +)$, $H = (3\mathbb{Z}, +)$, and $K = (2\mathbb{Z}$, +). 
        Then $H$ and $K$ are both subgroups of $G$, but $H \cup K$ is not, as it is not 
        closed under the group operation. For instance, $2, 3 \in G \cup H$ but $2 + 3 
        = 5 \notin G \cup H$.

    \end{enumerate}
\end{solution}

\begin{problem}{8}
    Let $G$ be a group, and fix an element $x \in G$. Define a function
    \begin{align*}
        \lambda_x: \; &G \rightarrow G\\
        &g \mapsto x^{-1}gx
    \end{align*}
    
    \begin{enumerate}[(a)]
        \item Prove that for all $x \in G$, $\lambda_x$ is a permutation of $G$.
        \item Prove that $G$ is abelian if and only if $\lambda_x = e$ for
        all $x \in G$.
    \end{enumerate}
\end{problem}

\begin{solution}
        \begin{enumerate}[(a)]
            \item \emph{Proof:} Let $x \in G$. Define $\lambda_x^{-1}(g) = xgx^{-1}$. Note this function
            is the inverse of $\lambda_x$: Let $g \in G$. Then $\lambda_x^{-1}(\lambda_x(g)) = \lambda_x^{-1}(x^{-1}gx) =
            x(x^{-1}gx)x^{-1} = xx^{-1}gxx^{-1} = ege = g$. Thus, as $\lambda_x$ has an inverse, and a function has an inverse
            if and only if it is a bijection, then $\lambda_x$ is a bijection from $G \rightarrow G$, and therefore a 
            permutation of $G$.

            \item Suppose $G$ is abelian and fix $x \in G$. Then $\lambda_x(g) = x^{-1}g{x} = x^{-1}xg = eg = g$,
            by G being abelian. Thus, $\lambda_x = e \in S_G$. For the other direction of the biconditional: Let $x, g
            \in G$ and suppose $\lambda_x = e \in S_G$. Thus, $\lambda_x(g) = g \Rightarrow x^{-1}gx = g \Rightarrow 
            x(x^{-1}gx) = xg \Rightarrow xx^{-1}gx = xg \Rightarrow egx = xg \Rightarrow gx = xg$. Thus as $x$ and $g$
            were any two arbitrary elements of $G$, $G$ is abelian.

        \end{enumerate}
\end{solution}

\end{document}