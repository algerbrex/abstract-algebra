\documentclass[12pt, letterpaper]{article}
\usepackage{amsfonts}
\usepackage{amssymb}
\usepackage{amsmath}
\usepackage{amsthm}
\usepackage{titling}
\usepackage{mathtools}
\usepackage{makecell}
\usepackage[shortlabels]{enumitem}
\usepackage[margin=1in]{geometry}

\setlength{\droptitle}{-8ex}
\pretitle{\begin{flushleft}\large}
\posttitle{\par\end{flushleft}}
\preauthor{\begin{flushleft}\large}
\postauthor{\end{flushleft}}
\predate{\begin{flushleft}\large}
\postdate{\end{flushleft}}

\title{Abstract Algebra (MATH-4620) TEST 1 TAKE-HOME COMPONENT}
\author{Christian Dean}
\date{September 1, 2023}


\newenvironment{problem}
    [1]
    {\noindent \textbf{Problem #1:}}
    {\vspace{3mm}}

\begin{document}

\maketitle

\noindent\hfil\rule{16cm}{0.4pt}\hfil

\begin{problem}{1}
    \begin{enumerate}[(a)]
        \item $\phi$ is a homomorphism: Let $a, b \in \mathbb{Q}^*$. Then $\phi(ab) = (ab)^2 = 
        abab = aabb = a^2b^2 = \phi(a)\phi(b)$. It's image is $\{\frac{p}{q} \in \mathbb{Q}^+ : 
        \exists \; k, l \in \mathbb{Z} \ni p = k^2 \land q = l^2\} = \{q^2 \;|\; q \in \mathbb{Q}^*\}$. 
        And it's kernal is $\{-1, 1\}$: We can solve the equation $q^2 = 1$: $q^2 = 1 \Rightarrow 
        q^{-1}qq = q^{-1}1 \Rightarrow q = q^{-1}$. And the only elements in the group $(\mathbb{Q},
        \cdot)$ that equal their inverses are $1$ and $-1$.

        \item $\phi$ is not a homomorphism: Let $a, b \in \mathbb{Z}$. Then $\phi(a + b) = 
        |a + b|$ and it is not true for any arbitrary $a$ and $b$ that $|a + b| = |a| + |b|
        = \phi(a) + \phi(b)$. For example $|-3 + 5|  = |2| = 2 \neq 8 = |-3| + |5|$.

        \item $\phi$ is a homomorphism: Let $a, b \in \mathbb{R}$. Then $\phi(a + b) = 
        e^{2\pi i(a + b)} = e^{2\pi ia + 2\pi ib} = e^{2\pi ia}e^{2\pi ib} = \phi(a)\phi(b)$.

        \bigskip\noindent
        The image of $\phi(\mathbb{R}) = U$. \emph{Proof}: Let $x \in \phi(\mathbb{R}) 
        \Rightarrow x = \text{cos}(2\pi r) + i\text{sin}(2\pi r)$, for some $r \in \mathbb{R}$. 
        By the pythagoren identity, $\text{cos}^2(2\pi r) + \text{sin}^2(2\pi r) = 1 \Rightarrow
        \text{the point } (\text{cos}(2\pi r), \\ \text{sin}(2\pi r)) \text{ is on the unit circle} 
        \Rightarrow x \in U$. Let $y \in U$. Then $y$ has the form $y = a + ib$, where $(a, b)$ is a 
        point on the unit circle. Every point on the unit circle has a unique corresponding angle, 
        modulo $2\pi$. Let $\alpha$ be the angle corresponding to $(a, b)$. Let $r = \frac{\alpha}{2\pi}$. 
        Then $\text{cos}(2\pi r) = \text{cos}(2\pi(\frac{\alpha}{2\pi})) = \text{cos}(\alpha) = a$ 
        and $\text{sin}(2\pi r) = \text{sin}(2\pi(\frac{\alpha}{2\pi})) = \text{sin}(\alpha) = b$. 
        Thus, there exists a real number $r$ such that $\text{cos}(2\pi r) + i\text{sin}(2\pi r) = 
        a + ib \Rightarrow y \in \phi(\mathbb{R})$. And as each set has been shown to be a subset
        of the other, the two sets are equal.

        \bigskip\noindent
        The kernal is $\mathbb{Z}$: The kernal will include any real number $r$ such that
        $\phi(r) = \text{cos}(2\pi r) + i\text{sin}(2\pi r) = 1 \Rightarrow \text{cos}(2\pi r)
        = 1 \text{ and } \text{sin}(2\pi r) = 0$. The angles that make cosine $1$ and sine $0$
        are the multiples of $2\pi$. Thus the kernal will be all real numbers that make multiples
        of $2\pi$, which are only the integers.

        \item $\mathbb{Q}^* / \{1, -1\} \cong \{q^2 \;|\; q \in \mathbb{Q}^*\}$ and $\mathbb{R}
        / \mathbb{Z} \cong U$.

    \end{enumerate}
\end{problem}

\begin{problem}{2}
    \begin{enumerate}[(a)]
        \item $|G| = 4(36) = 144$ and $H = \{(0, 0), (1, 9), (2, 18), (3, 27)\} \Rightarrow |H| = 4$.
        So $|G/H| = |G|/|H| = 144 / 4 = 36$.

        \item The order of $\overline{(2,6)}$ will be the smallest $n \in N$ such that $(2,6)^n =
        n(2, 6) = (2n, 6n) \in H$. Thus $n = 3$, as $(2,6)^1 = (2, 6)$, $(2,6)^2 = (0, 12)$, 
        and $(2, 6)^3 = (2, 18) \in H$.

        \item 
        \begin{enumerate}[1.]
            \item $\mathbb{Z}_2 \times \mathbb{Z}_2 \times \mathbb{Z}_3 \times \mathbb{Z}_3$
            \item $\mathbb{Z}_2 \times \mathbb{Z}_2 \times \mathbb{Z}_9$
            \item $\mathbb{Z}_4 \times \mathbb{Z}_3 \times \mathbb{Z}_3$
            \item $\mathbb{Z}_4 \times \mathbb{Z}_9$
        \end{enumerate}

        \item $G/H$ is cyclic and can be generated by the element $(1, 2)$:
        \begin{align*}
            1 \cdot (1, 2) &= (1, 2)\\
            2 \cdot (1, 2) &= (2, 4)\\
            3 \cdot (1, 2) &= (3, 6)\\
            4 \cdot (1, 2) &= (0, 8)\\
            6 \cdot (1, 2) &= (2, 12)\\
            9 \cdot (1, 2) &= (1, 18)\\
            12 \cdot (1, 2) &= (0, 24)\\
            18 \cdot (1, 2) &= (2, 18)\\
            36 \cdot (1, 2) &= (0, 0)
        \end{align*}
        Thus $G/H$ must be isomorphic to $\mathbb{Z}_4 \times \mathbb{Z}_9$, as $\mathbb{Z}_4 \times 
        \mathbb{Z}_9$ is the only cyclic group in the set of isomorphism classes.
    \end{enumerate}
\end{problem}

\begin{problem}{3}
    \begin{enumerate}[(a)]
        \item By \textbf{Theorem 55}, $\mathbb{Z}_{31}$ is a field. It had characteristic $31$ and 
        the set of units are all the elements in $\mathbb{Z}_{31}$ but $0$.

        \item $\mathbb{Z}_{33}$ is a commutative rung, but not an integral domain, as it has zero 
        divisors (e.g. $3 \cdot 11 = 0 (\text{mod } 33)$). It has characteristic $33$ and the set 
        of units are all the elements in $\mathbb{Z}_{33}$ that are relatively prime to $33$: 
        $\mathbb{Z}_{33} \setminus \{3, 11\}$.

        \item Let $R$ be the algebraic structure in question. $R$ is a commutative ring, but is
        not a rung: Suppose there existed a multiplicative identity in $R$, denoted $e + ie$. Let
        $x + ix \in R$ such that $x \neq 0$. Then:
        \begin{align*}
            &(x + ix)(e + ie) = x + ix\\
            \Rightarrow \; &xe + ixe + ixe -xe = x + ix\\
            \Rightarrow \; &i(2xe) = x + ix
        \end{align*}
        And this is a contradiction, as a purely imaginary number cannot equal a complex number
        with a non-zero real part. Thus $R$ is not a rung, and thus not an integral domain. 
        The characteristic is $0$, as adding any non-zero element of $R$ will never get you back 
        to $0$.

        \item By \textbf{Theorem 67}, the ring $\mathbb{Z}[x]$ is an integral domain with characteristic
        $0$, as $\mathbb{Z}$ is an integral domain with characteristic $0$. The set of units is exactly
        the set of units of $Z$: $\{1, -1\}$.

        \item The ring $2\mathbb{Z}[x]$ is an commutative ring with characteristic $0$, as $2\mathbb{Z}$ 
        is a commutative ring with characteristic $0$. It is not a rung though, as from observation no 
        element of $2\mathbb{Z}$ can serve as a multiplicative identity. Thus, $2\mathbb{Z}[x]$ is also
        not an integral domain.

    \end{enumerate}
\end{problem}

\begin{problem}{4}
    \begin{enumerate}[(a)]
        \item Let $R$ denote the algebraic structure in question. $R$ is not a ring, as the
        distributive property fails. Consider $f(x) = x^2, g(x) = 2x^3, h(x) = 5x^3 \in R$.
        Then $f(x) \circ (g(x) + h(x)) = f(g(x) + h(x)) = f(2x^3 + 5x^3) = f(7x^3) = 
        (7x^3)^2 = 49x^6$. But $f(x) \circ g(x) + f(x) \circ h(x) = f(g(x)) + f(h(x)) = 
        f(2x^3) + f(5x^3) = (2x^3)^2 + (5x^3)^2 = 4x^6 + 25x^6 = 29x^6$.

        \item  $F \times F$ is a commutative rung: Let $G$ be the underlining abelian group of 
        $F$. Then $F \times F$ induces an abelian group $G \times G$. Multiplication is associative
        and distrbutes over addition in $F x F$, as $F$ is a field. And the multiplicative identity
        is $(1_F, 1_F)$.

        \bigskip\noindent
        $F \times F$ is not an integral domain however, as it has zero divisors: $(1_F, 0_F) \cdot
        (0_F, 1_F) = (0_F, 0_F)$.
        
        \item 
    \end{enumerate}
\end{problem}

\end{document}