\documentclass[12pt, letterpaper]{article}
\usepackage{amsfonts}
\usepackage{amssymb}
\usepackage{amsmath}
\usepackage{amsthm}
\usepackage{titling}
\usepackage{mathtools}
\usepackage{makecell}
\usepackage[shortlabels]{enumitem}
\usepackage[margin=1in]{geometry}

\setlength{\droptitle}{-8ex}
\pretitle{\begin{flushleft}\large}
\posttitle{\par\end{flushleft}}
\preauthor{\begin{flushleft}\large}
\postauthor{\end{flushleft}}
\predate{\begin{flushleft}\large}
\postdate{\end{flushleft}}

\title{Abstract Algebra (MATH-4620) TEST 1 TAKE-HOME COMPONENT}
\author{Christian Dean}
\date{September 1, 2023}


\newenvironment{problem}
    [1]
    {\noindent \textbf{Problem #1:}}
    {\vspace{3mm}}

\begin{document}

\maketitle

\noindent\hfil\rule{16cm}{0.4pt}\hfil

\begin{problem}{1}
    \begin{enumerate}[(a)]
        \item $\phi$ is a homomorphism: Let $a, b \in \mathbb{Q}^*$. Then $\phi(ab) = (ab)^2 = 
        abab = aabb = a^2b^2 = \phi(a)\phi(b)$. It's image is $\{\frac{p}{q} \in \mathbb{Q}^+ : 
        \exists \; k, l \in \mathbb{Z} \ni p = k^2 \land q = l^2\} = \{q^2 \;|\; q \in \mathbb{Q}^*\}$. 
        And it's kernal is $\{-1, 1\}$: We can solve the equation $q^2 = 1$: $q^2 = 1 \Rightarrow 
        q^{-1}qq = q^{-1}1 \Rightarrow q = q^{-1}$. And the only elements in the group $(\mathbb{Q},
        \cdot)$ that equal their inverses are $1$ and $-1$.

        \item $\phi$ is not a homomorphism: Let $a, b \in \mathbb{Z}$. Then $\phi(a + b) = 
        |a + b|$ and it is not true for any arbitrary $a$ and $b$ that $|a + b| = |a| + |b|
        = \phi(a) + \phi(b)$. For example $|-3 + 5|  = |2| = 2 \neq 8 = |-3| + |5|$.

        \item $\phi$ is a homomorphism: Let $a, b \in \mathbb{R}$. Then $\phi(a + b) = 
        e^{2\pi i(a + b)} = e^{2\pi ia + 2\pi ib} = e^{2\pi ia}e^{2\pi ib} = \phi(a)\phi(b)$.

        \bigskip\noindent
        The image of $\phi(\mathbb{R}) = U$. \emph{Proof}: Let $x \in \phi(\mathbb{R}) 
        \Rightarrow x = \text{cos}(2\pi r) + i\text{sin}(2\pi r)$, for some $r \in \mathbb{R}$. 
        By the pythagoren identity, $\text{cos}^2(2\pi r) + \text{sin}^2(2\pi r) = 1 \Rightarrow
        \text{the point } (\text{cos}(2\pi r), \\ \text{sin}(2\pi r)) \text{ is on the unit circle} 
        \Rightarrow x \in U$. Let $y \in U$. Then $y$ has the form $y = a + ib$, where $(a, b)$ is a 
        point on the unit circle. Every point on the unit circle has a unique corresponding angle, 
        modulo $2\pi$. Let $\alpha$ be the angle corresponding to $(a, b)$. Let $r = \frac{\alpha}{2\pi}$. 
        Then $\text{cos}(2\pi r) = \text{cos}(2\pi(\frac{\alpha}{2\pi})) = \text{cos}(\alpha) = a$ 
        and $\text{sin}(2\pi r) = \text{sin}(2\pi(\frac{\alpha}{2\pi})) = \text{sin}(\alpha) = b$. 
        Thus, there exists a real number $r$ such that $\text{cos}(2\pi r) + i\text{sin}(2\pi r) = 
        a + ib \Rightarrow y \in \phi(\mathbb{R})$. And as each set has been shown to be a subset
        of the other, the two sets are equal.

        \bigskip\noindent
        The kernal is $\mathbb{Z}$: The kernal will include any real number $r$ such that
        $\phi(r) = \text{cos}(2\pi r) + i\text{sin}(2\pi r) = 1 \Rightarrow \text{cos}(2\pi r)
        = 1 \text{ and } \text{sin}(2\pi r) = 0$. The angles that make cosine $1$ and sine $0$
        are the multiples of $2\pi$. Thus the kernal will be all real numbers that make multiples
        of $2\pi$, which are only the integers.

        \item $\mathbb{Q}^* / \{1, -1\} \cong \{q^2 \;|\; q \in \mathbb{Q}^*\}$ and $\mathbb{R}
        / \mathbb{Z} \cong U$.

    \end{enumerate}
\end{problem}

\begin{problem}{2}
    \begin{enumerate}[(a)]
        \item $|G| = 4(36) = 144$ and $H = \{(0, 0), (1, 9), (2, 18), (3, 27)\} \Rightarrow |H| = 4$.
        So $|G/H| = |G|/|H| = 144 / 4 = 36$.

        \item The order of $\overline{(2,6)}$ will be the smallest $n \in N$ such that $(2,6)^n =
        n(2, 6) = (2n, 6n) \in H$. Thus $n = 3$, as $(2,6)^1 = (2, 6)$, $(2,6)^2 = (0, 12)$, 
        and $(2, 6)^3 = (2, 18) \in H$.

        \item 
        \begin{enumerate}[1.]
            \item $\mathbb{Z}_2 \times \mathbb{Z}_2 \times \mathbb{Z}_3 \times \mathbb{Z}_3$
            \item $\mathbb{Z}_2 \times \mathbb{Z}_2 \times \mathbb{Z}_9$
            \item $\mathbb{Z}_4 \times \mathbb{Z}_3 \times \mathbb{Z}_3$
            \item $\mathbb{Z}_4 \times \mathbb{Z}_9$
        \end{enumerate}

        \item $G/H$ is cyclic and can be generated by the element $\overline{(1, 2)}$:
        \begin{align*}
            1 \cdot (1, 2) &= (1, 2)\\
            2 \cdot (1, 2) &= (2, 4)\\
            3 \cdot (1, 2) &= (3, 6)\\
            4 \cdot (1, 2) &= (0, 8)\\
            6 \cdot (1, 2) &= (2, 12)\\
            9 \cdot (1, 2) &= (1, 18)\\
            12 \cdot (1, 2) &= (0, 24)\\
            18 \cdot (1, 2) &= (2, 18)\\
            36 \cdot (1, 2) &= (0, 0)
        \end{align*}
        Thus $G/H$ must be isomorphic to $\mathbb{Z}_4 \times \mathbb{Z}_9$, as $\mathbb{Z}_4 \times 
        \mathbb{Z}_9$ is the only cyclic group in the set of isomorphism classes.
    \end{enumerate}
\end{problem}

\begin{problem}{3}
    \begin{enumerate}[(a)]
        \item \textbf{Field?} Yes; By \textbf{Theorem 55}, $\mathbb{Z}_{31}$ is a field, and thus is 
        also a ring, rung, commutative ring, commutative rung, and integral domain. \textbf{Characteristic}: 
        $31$. \textbf{Units}: $\mathbb{Z}_{31} \setminus \{0\}$.

        \item \textbf{Commutative rung?} Yes; $(\mathbb{Z}_{33}, +_{33})$ is an additive abelian group.
        Normal multiplication among real numbers is associative, and the distributive property holds 
        among real numbers. Thus it is also a ring, rung, and commutative ring. \textbf{Integral domain?}
        No; $3$ and $11$ are zero divisors. Thus $\mathbb{Z}_{33}$ is also not a field. 
        \textbf{Characteristic}: $33$. \textbf{Units}: $\mathbb{Z}_{33} \setminus \{3, 11\}$.

        \item Let $R$ be the algebraic structure in question. \textbf{Commutative ring?} Yes;
        The described set of complex numbers forms an abelian group under complex number addition.
        Multiplication is associative among complex numbers, and the distributive property holds for
        complex numbers. Thus $R$ is also a ring. 
        
        \bigskip\noindent
        \textbf{Rung?} No; Suppose there existed a multiplicative identity in $R$, denoted $e + ie$. 
        Let $x + ix \in R$ such that $x \neq 0$. Then:

        \begin{align*}
            &(x + ix)(e + ie) = x + ix\\
            \Rightarrow \; &xe + ixe + ixe -xe = x + ix\\
            \Rightarrow \; &i(2xe) = x + ix
        \end{align*}

        And this is a contradiction, as a purely imaginary number cannot equal a complex number
        with a non-zero real part. Thus $R$ is not a rung, and also not a commutative rung, integral
        domain, or field.

        \bigskip\noindent
        \textbf{Characteristic}: $0$. \textbf{Units:} N/A.

        \item \textbf{Integral domain?} Yes; by \textbf{Theorem 67}, the ring $\mathbb{Z}[x]$ is an integral 
        domain, as $\mathbb{Z}$ is an integral domain. Thus it is also a ring, rung, commutative ring, and 
        commutative rung. \textbf{Field?} No; Non-constant polynomials will not have multiplicative inverses
        as multiplying them by any other polynomial will not decrease their degree. \textbf{Characteristic}:
        $0$, as the characteristic of $\mathbb{Z}$ is $0$. \textbf{Units}: The set of units is exactly the set 
        of units of $Z$: $\{1, -1\}$.

        \item \textbf{Commutative ring?} Yes; The ring $2\mathbb{Z}[x]$ is an commutative ring as $2\mathbb{Z}$ 
        is a commutative ring. Thus it is also ring. \textbf{Rung?} No, as from observation no element of $2\mathbb{Z}$ 
        can serve as a multiplicative identity. Thus it is also not a commutative rung, integral domain, or field.
        \textbf{Characteristic}: $0$, as the characteristic of $2\mathbb{Z}$ is $0$. \textbf{Units}: N/A.

    \end{enumerate}
\end{problem}

\begin{problem}{4}
    \begin{enumerate}[(a)]
        \item Let $R$ denote the algebraic structure in question. \textbf{Ring?} No, as the
        distributive property fails. Consider $f(x) = x^2, g(x) = 2x^3, h(x) = 5x^3 \in R$.
        Then $f(x) \circ (g(x) + h(x)) = f(g(x) + h(x)) = f(2x^3 + 5x^3) = f(7x^3) = 
        (7x^3)^2 = 49x^6$. But $f(x) \circ g(x) + f(x) \circ h(x) = f(g(x)) + f(h(x)) = 
        f(2x^3) + f(5x^3) = (2x^3)^2 + (5x^3)^2 = 4x^6 + 25x^6 = 29x^6$.

        \bigskip\noindent
        Thus, $R$ is also not a rung, commutative ring, commutative rung, integral domain, or
        field. The characteristic or set of units also cannot be considered.

        \item  
        \textbf{Commutative rung?} Yes; Let $G$ be the underlining abelian group of $F$. Then $F 
        \times F$ induces an abelian group $G \times G$. Multiplication is associative and distrbutes 
        over addition in $F \times F$, as $F$ is a field. And the multiplicative identity is $(1_F, 1_F)$.
        Thus, $F \times F$ is also a ring, rung, and commutative ring.

        \bigskip\noindent
        \textbf{Integral domain?} No; $(1_F, 0_F)$ and $(0_F, 1_F)$ are zero divisors. Thus, $R$ is also
        not a field.

        \bigskip\noindent
        \textbf{Characteristic}: $13$, as the entry in any given pair in $F \times F$ will have characteristic
        $13$. Thus if we add any given pair to itself $13$ times, each entry in the pair will be $0_F$. 
        \textbf{Units:} $(F \times F)^{\times} = \{(a, b) \in F \times F \;|\; a \neq 0_F \text{ and } b \neq 0_F\}$.
        
        \item  \textbf{Field} Yes. First, $(X, +)$ is an abelian group. The identity is $0$. Every element is 
        its own additive inverse.

        \bigskip\noindent
        And $+$ is associative. Note for any $x, y, z \in X$, if any of $x$, $y$, or $z$ are $0$, the
        test for associativity becomes a test of commutativity. And $+$ is defined to be commutative, cases where
        $x$, $y$, or $z$ equal $0$ are associative. Thus, there are four cases left to consider: $x = y$, $y = z$, 
        $x = z$, or $x, y, z$ are pair-wise distinct.
        
        \bigskip\noindent
        \emph{Proof}: Suppose $x, y, z \in X \setminus \{0\}$. Let $Y = X \setminus \{0\}$. \textbf{Case 1}: $x = y$. 
        Let $n$ denote the element in $Y$ that is not equal to $x = y$ or $z$. Then $(x + y) + z = (y + y) + z = 0 + z 
        = z = y + n = y + (y + z) = x + (y + z)$. \textbf{Case 2}: $y = z$. Let $n$ denote the element in $Y$ that is not
        equal to $y = z$ or $x$. Then $x + (y + z) = x + (y + y) = x + 0 = x = n + y = (x + y) + y = (x + y) + z$. 
        \textbf{Case 3}: $x = z$. Let $n$ denote the element in $Y$ that is not equal to $x = z$ or $y$. Then $(x + y) + 
        z = (z + y) + z = n + z = z + n = z + (y + z) = x + (y + z)$. \textbf{Case 4} $x, y, z$ are pair-wise distinct. 
        Then $(x + y) + z = z + z = 0 = x + x = x + (y + z)$.

        \bigskip\noindent
        Addtionally, $\times$ is associative. Note for any $x, y, z \in X$, if $x$, $y$ or $z$ equal $0$ then the test of
        associativity becomes a test of commutativity. And as $\times$ is defined to commutative, cases where $x$, $y$, or $z$ 
        equal $1$ are associative. Further, if any of $x$, $y$, $z$ are $0$, then $(xy)z = 0$ and $x(yz) = 0$, so cases where
        $x$, $y$, or $z$ equal $0$ are associative. This leaves eight cases where the only values $x, y, z$ take on are $a$ or $b$.
        But only four need to be checked due to commutative:

        \begin{align*}
            &(aa)b = bb = a = a(ab)\\
            &(bb)a = aa = b = b1 = b(ba)\\
            &(aa)a = ba = ab = a(aa)\\
            &(bb)b = ab = 1 = ba = b(bb)\\
            (ba)a = &(ab)a = 1a = a1 = a(ba)\\
            (ba)b = &(ab)b = 1b = b = aa = a(bb)
        \end{align*}

        Further, the distributive property holds. Note for any $x, y, z \in X$, where we want to show $x(y + z) = xy + xz$,
        if $x$, $y$ or $z$ equal $0$ then the distributive property test becomes a test of the commutativity of multiplication. 
        And as multiplication is defined to be commutative, the distributive property holds in these cases. Further, if $x = 1$,
        then $x(y + z) = y + z = 1y + 1z = xy + xz$, and thus the distributive property holds in this case. Also note if $y = z$,
        then $x(y + z) = x(y + y) = x0 = 0 = 1 + 1 = xy + xy = xy + xz$. Thus in this case the distributive property holds as well.
        This leaves $12$ cases left to check: When $x$ equals $a$ or $b$ (2 choices), $y$ equals $a$, $b$, or $c$ (3 choices), and 
        $z$ equals whatever $y$ does not equal (2 choices). However, due to commutativity, only six cases need to be checked:

        \begin{align*}
            a(b + a) = a(a + b) = a(1) = b + 1  = aa + ab\\
            a(1 + a) = a(a + 1) = a(b) = 1 = b + a = aa + a\\
            a(1 + b) = a(b + 1) = a(a) = b = 1 + a = ab + a\\
            b(b + a) + b(a + b) = b(1) = 1 + a = ba + bb \\
            b(1 + a) = b(a + 1) = b(b) = a = 1 + b = ba + b\\
            b(1 + b) = b(b + 1) = b(a) = 1 = a + b = bb + b
        \end{align*}

        Further $\times$ is defined to be commutative, $1$ is the multiplicative identity, and every non-zero element has a 
        multiplicative inverse: $(1)(1) = 1$, $ab = ba = 1$. Thus $(X, +, \times)$ is a field, and also a ring, rung, commutative 
        ring, commutative rung, and integral domain.
    \end{enumerate}
\end{problem}

\end{document}