\documentclass[12pt, letterpaper]{article}
\usepackage{amsfonts}
\usepackage{amssymb}
\usepackage{amsmath}
\usepackage{amsthm}
\usepackage{titling}
\usepackage{mathtools}
\usepackage{multicol}
\usepackage[shortlabels]{enumitem}
\usepackage[margin=1in]{geometry}

\setlength{\droptitle}{-8ex}
\pretitle{\begin{flushleft}\large}
\posttitle{\par\end{flushleft}}
\preauthor{\begin{flushleft}\large}
\postauthor{\end{flushleft}}
\predate{\begin{flushleft}\large}
\postdate{\end{flushleft}}

\title{Abstract Algebra (MATH-4620) Test \#1 Notes Summary}
\author{Christian Dean}

\begin{document}

\maketitle

\noindent\hfil\rule{16cm}{0.4pt}\hfil
  
\begin{multicols}{2}
    \begin{itemize}
        \item What is abstract algebra
        \item Symmetries of the square
        \item Binary operator
        \item Properties of binary operators
        
        \item (\textbf{THRM}) Binary operator has at most one identity element 
        
        \item Isomorphic and homomorphism
        
        \item Properties preserved under isomorphism
        
        \item Group
        
        \item Exponet notation of groups
        
        \item Subgroup
        
        \item (\textbf{THRM}) If $(S, *)$ and $(T, \#)$ are binary structures, and $\phi: S \rightarrow T$ is
        an isomorphism, then $\phi^{-1}: T \rightarrow S$ is an isomorphism 

        \item Isomorphisms are equivelence
        relations that create paritions among sets of groups

        \item (\textbf{THRM}) Every subgroup of $(\mathbb{Z}, +)$ takes the form $n\mathbb{Z}$
        for some $n \in \mathbb{Z}$

        \item Cyclic groups, subgroups, and cyclic groups are countable
        
        \item Order of a group
        
        \item (\textbf{THRM}) Suppose $G$ is a group and $a \in G$. If $|a|$ is finite,
        then $|a| = \text{ the smallest } n \in \mathbb{N} \ni a^n = e$

        \item (\textbf{THRM}) Suppose $G$ is a cyclic group. If $|G| = n, n \in \mathbb{N}$, then
        $G \cong \mathbb{Z}_n$. Or if $G$ is infinite, $G \cong \mathbb{Z}$
        
        \item All subgroups of infinite cyclic groups are cyclic
      
        \item (\textbf{THRM}) On the subgroups $Z_n, n \in \mathbb{N}$ has
        
        \item Permutation groups
      
        \item (\textbf{THRM}) Suppose $X$ and $Y$ are sets. If $|X| = |Y|$ (there exists a bijection between them), 
        then $S_X \cong S_Y$
        
        \item Symmetry and diheadral groups (denoted $S_n$ and $D_n$) and their orders
        
        \item (\textbf{THRM}) $S_n$ and $D_n$ are non-abelian
        
        \item Orbits and cycles
        
        \item Two-line vs cyclic notation of permutations
        
        \item Caley's theorem
      
        \item Order of permutation ${<}\sigma{>}, \sigma \in S_n$
        
        \item Disjoint vs joint cycles
      
        \item Transpositions (2-cycles)
        
        \item (\textbf{THRM}) Every cycle can be written as a product of transpositions,
        which implies every permutation can be written as a product of transpositions
      
        \item (\textbf{THRM}) If $n \in \mathbb{N}$ and $\sigma \in S_n$, then $\sigma$ can be 
        written as a product of an even or odd number of transpositions, but not both
        
        \item If $\sigma \in S_n$ is a product of an even number of transpositions, we call it an even 
        permutation, and likewise for odd
      
        \item Exactly half of all permutatioms in $S_n$ are even and half are odd
    \end{itemize}

\end{multicols}

\end{document}