\documentclass[12pt, letterpaper]{article}
\usepackage{amsfonts}
\usepackage{amssymb}
\usepackage{amsmath}
\usepackage{amsthm}
\usepackage{titling}
\usepackage[margin=1in]{geometry}

\setlength{\droptitle}{-8ex}
\pretitle{\begin{flushleft}\large}
\posttitle{\par\end{flushleft}}
\preauthor{\begin{flushleft}\large}
\postauthor{\end{flushleft}}
\predate{\begin{flushleft}\large}
\postdate{\end{flushleft}}

\title{Abstract Algebra (MATH-4620) HOMEWORK 1 Solutions}
\author{Christian Dean}
\date{September 1, 2023}


\newenvironment{problem}
    [1]
    {\noindent \textbf{Problem #1.}}
    {\vspace{3mm}}


\newenvironment{solution}
    [0]
    {\noindent \textbf{Solution:}} 
    {\vspace{3mm}}


\begin{document}

\maketitle

\noindent\hfil\rule{16cm}{0.4pt}\hfil

\section{Groups and Subgroups (pg. 26)}
    \begin{problem}{7}
        Define $*$ on $\mathbb{Z}$ by letting $a*b=a-b$. Is $*$ commutative? 
        associative?
    \end{problem}

    \begin{solution}
        $*$ is not commutative or associative as regular subtraction on
        $\mathbb{R}$ is not commutative or associative.
    \end{solution}

    \begin{problem}{8}
        Define $*$ on $\mathbb{Q}$ by letting $a*b=ab+1$. Is $*$ commutative?
        associative?
    \end{problem}

    \begin{solution}
        $*$ is commutative as regular multiplication on $\mathbb{Q}$ is commutative.
        But it is not associative. Consider $a=2, b=5, c=-3$. Then $a * (b * c) = 
        2 * (5 * -3) = -27$. But $(a * b) * c = (2 * 5) * -3 = -32$.
    \end{solution}

    \begin{problem}{9}
        Define $*$ on $\mathbb{Q}$ by letting $a*b=ab/2$. Is $*$ commutative?
        associative?
    \end{problem}

    \begin{solution}
        $*$ is commutative as regular multiplication on $\mathbb{Q}$ is commutative.
        And it is also associative. \emph{Proof}: Let $a,b,c \in \mathbb{Q}$. Then
        $(a * b) * c = \frac{1}{2}ab * c = \frac{1}{2}(\frac{1}{2}ab)(c) = \frac{1}{4}abc$. 
        And, $a * (b * c) = a * \frac{1}{2}bc = \frac{1}{2}(a)(\frac{1}{2}bc) = \frac{1}{4}abc$.
    \end{solution}

    \begin{problem}{10}
        Define $*$ on $\mathbb{Z}^+$ by letting $a*b=2^{ab}$. Is $*$ commutative?
        associative?
    \end{problem}

    \begin{solution}
        $*$ is commutative as regular multiplication on $\mathbb{Q}$ is commutative.
        But it is not associative. Consider $a=1, b=2, c=3$. Then $a * (b * c) = 
        1 * (2 * 3) = 1 * 2^6 = 2^{64}$. But $(a * b) * c = (1 * 2) * 3 = 2^2 * 3 =
        2^{12}$.
    \end{solution}

    \begin{problem}{11}
        Define $*$ on $\mathbb{Z}^+$ by letting $a*b=a^b$. Is $*$ commutative?
        associative?
    \end{problem}

    \begin{solution}
        $*$ is not commutative. Consider $a=3,b=4$. Then $a * b = 3 * 4 = 3^4 = 81$,
        but $b * a = 4 * 3 = 4^3 = 64$. It is also not associative. Consider 
        $a=2, b=3, c=4$. Then $a * (b * c) = 2 * (3 * 4) = 2 * 81 = 2^{81}$. 
        But $(a * b) * c = (2 * 3) * 4 = 8 * 4 = 8^4 = 2^{12}$.
    \end{solution}

\section{Groups and Subgroups (pg. 34)}
    \begin{problem}{2}
        Is $(\mathbb{Z}, +)$ isomorphic to $(\mathbb{Z}, +)$ where $\phi(n)=-n, \forall n \in \mathbb{Z}$?
    \end{problem}

    \begin{solution}
        Yes. \emph{Proof:} Let $x, y \in \mathbb{Z}$. Then $\phi(x + y)=-x-y$ and $\phi(x)+\phi(y)=-x-y$.
        and $\phi$ is a bijection as $\phi(0) = 0$, and every other integer is mapped to exactly one integer,
        its additive inverse.
    \end{solution}

    \begin{problem}{3}
        Is $(\mathbb{Z}, +)$ isomorphic to $(\mathbb{Z}, +)$ where $\phi(n)=2n, \forall n \in \mathbb{Z}$?
    \end{problem}

    \begin{solution}
        No, as $\phi$ is not surjective, and thus not bijective. \emph{Proof:} Let $y = 1$.
        But $\nexists x \in \mathbb{Z} \ni \phi(x) = y = 1$.
    \end{solution}

    \begin{problem}{4}
        Is $(\mathbb{Z}, +)$ isomorphic to $(\mathbb{Z}, +)$ where $\phi(n)=n + 1, \forall n \in \mathbb{Z}$?
    \end{problem}

    \begin{solution}
        No. \emph{Proof:} let $x, y \in \mathbb{Z}$. Then $\phi(x + y)= x + y + 1$. But $\phi(x)+\phi(y)=
        (x + 1) + (y + 1) = x + y + 2$.
    \end{solution}

    \begin{problem}{5}
        Is $(\mathbb{Q}, +)$ isomorphic to $(\mathbb{Q}, +)$ where $\phi(x)=x/2, \forall x \in \mathbb{Q}$?
    \end{problem}

    \begin{solution}
        Yes. $\phi$ is a bijection. \emph{Proof:} Let $x, y \in \mathbb{Q}$. Suppose $\phi(x) = \phi(y)$.
        Then $x/2 = y/2 \implies x = y$. And, let $y \in \mathbb{Q}$. Choose $x = 2y$. Then $\phi(x) = 
        \phi(2y) = y$. And it has the homomorphic property. \emph{Proof:} Let $x, y \in \mathbb{Z}$. 
        Then $\phi(x + y)=(x+y)/2$ and $\phi(x)+\phi(y)=x/2+y/2=(x+y)/2$.
    \end{solution}

    \begin{problem}{6}
        Is $(\mathbb{Q}, \cdot)$ isomorphic to $(\mathbb{Q}, \cdot)$ where $\phi(x)=x^2, \forall x \in \mathbb{Q}$?
    \end{problem}

    \begin{solution}
        No, since $\phi$ is not injective or surjective.
    \end{solution}

    \begin{problem}{7}
        Is $(\mathbb{R}, \cdot)$ isomorphic to $(\mathbb{R}, \cdot)$ where $\phi(x)=x^3, \forall x \in \mathbb{R}$?
    \end{problem}

    \begin{solution}
        No. Let $x, y \in \mathbb{R}$. Then $\phi(x + y) = (x+y)^3$, but $\phi(x)+\phi(y)=x^3+y^3$
    \end{solution}
\section{Groups and Subgroups (pg. 36)} 
    \begin{problem}{26}
    \end{problem}

    \begin{solution}
    \end{solution}
\section{Groups and Subgroups (pg. 42)}
    \begin{problem}{8}
        Give the table for the group for the set ${1, 3, 5, 7}$ with multiplication
        $\cdot_8$ modulo 8.
    \end{problem}

    \begin{solution}
        \begin{center}
        \begin{tabular}{ c | c c c c } 
            $\cdot_8$ & 1 & 3 & 5 & 7 \\ 
            \hline
            1 & 1 & 3 & 5 & 7 \\ 
            3 & 3 & 1 & 7 & 5 \\ 
            5 & 5 & 7 & 1 & 3 \\ 
            7 & 7 & 5 & 3 & 1 \\
        \end{tabular}
        \end{center}
    \end{solution}
\section{Non-book problems}
    \begin{problem}{5a}
        Prove if $f:X \rightarrow Y$ and $g: Y \rightarrow Z$ are both injections,
        then $g \circ f: X \rightarrow Z$ is also an injection.
    \end{problem}

    \begin{solution}
        \emph{Proof}: Let $a, b \in X$, and suppose $g \circ f(a) = g \circ f(b)$. As $g$
        is an injection, then $f(a) = f(b)$. And as $f$ is an injection, $a = b$. Thus,
        $g \circ f(a)$ is an injection.
    \end{solution}

    \begin{problem}{5b}
        Prove if $f:X \rightarrow Y$ and $g: Y \rightarrow Z$ are both surjections,
        then $g \circ f: X \rightarrow Z$ is also a surjection.
    \end{problem}

    \begin{solution}
        \emph{Proof}: Let $z \in Z$. As $g$ is a surjection, $\exists y \in Y \ni g(y)=z$.
        And as $f$ is a surjection, $\exists x \in X \ni f(x)=y$. Therefore, $z$ has a 
        pre-image in $X$, $x$, and $g \circ f(x)=z$. Thus, $g \circ f$ is a surjection.
    \end{solution}

    \begin{problem}{5c}
        Prove if $f:X \rightarrow Y$ and $g: Y \rightarrow Z$ are both bijections,
        then $g \circ f: X \rightarrow Z$ is also a bijection.
    \end{problem}

    \begin{solution}
        \emph{Proof}: Since $f$ and $g$ are bijections, by defintion, they're both
        injections. And by the proof given in 5a, $g \circ f$ is also an injection.
        And similarly since $f$ and $g$ are bijections, by defintion, they're both
        surjections. And by the proof given in 5b, $g \circ f$ is also a surjection.
        Thus, since $g \circ f$ is both an injection and a surjection, by defintion
        it is a bijection.
    \end{solution}

    \begin{problem}{6}
        Let $\mathcal{F}(\mathbb{R})$ be the set of all functions from $\mathbb{R}$ to 
        $\mathbb{R}$, and consider the binary operation of composition on
        $\mathcal{F}(\mathbb{R})$. Is this operation commutative?
    \end{problem}

    \begin{solution}
        No. \emph{Proof:} Let $f(x)=5x^2$ and $g(x)=2x+5$. Then $g \circ f = 2(5x^2)+5 = 
        10x^2 + 5$, but $f \circ g = 5(2x+5)^2 = 20x^2+100x+125$.
    \end{solution}

    \begin{problem}{7}
        Let $\mathcal{C}^\infty(\mathbb{R})$ be the set of all functions
        $\mathbb{R} \rightarrow \mathbb{R}$ which are infintely differentiable.
        Consider $\phi: \mathbb{R} \rightarrow \mathbb{R}, \phi(f) = f'$. Is
        $\phi$ an isomorphism from $(\mathcal{C}^\infty(\mathbb{R}), +)$ to
        $(\mathcal{C}^\infty(\mathbb{R}), +)$?
    \end{problem}

    \begin{solution}
        No, since $\phi$ is not a bijection. \emph{Proof:} Consider any two 
        polynomial functions that differ only by a constant. They're both
        infintely differentiable and will both be mapped to the same derivative
        by $\phi$.
    \end{solution}

    \begin{problem}{8}
        Let $S = {a, b, c}$ and consider the partial table:
        \begin{center}
        \begin{tabular}{ c | c c c } 
            $\star$ & a & b & c \\ 
            \hline
            a & b & c & a \\ 
            b & c & $\square$ & $\square$ \\ 
            c & a & b & $\square$ \\ 
        \end{tabular}
        \end{center}
    \end{problem}

    \begin{problem}{8a}
        Fill in the empty spaces in such a way that $\star$ becomes and associative binary 
        operation.
    \end{problem}

    \begin{solution}
        \begin{center}
            \begin{tabular}{ c | c c c } 
                $\star$ & a & b & c \\ 
                \hline
                a & b & c & a \\ 
                b & c & c & a \\ 
                c & a & b & b\\ 
            \end{tabular}
            \end{center}
    \end{solution}

    \begin{problem}{8b}
        Is the resulting operation commutative?
    \end{problem}

    \begin{solution}
        No. Consider $b \star c = a$ but $c \star b = b$.
    \end{solution}

    \begin{problem}{8b}
        Is $(S, \star)$ a group?
    \end{problem}

    \begin{solution}
        No, as there is no identiy element by observation of the table.
    \end{solution}

    \begin{problem}{9}
        Consider the set $S = {(x, y) \in \mathbb{R}^2 | x^2 + y^2 = 1}$. Define
        $\star$ by the following rule: $(x_1, y_1) \star (x_2, y_2) = 
        (x_1 x_2 - y_1 y_2, x_1 y_2 + x_2 y_1)$.
    \end{problem}

    \begin{problem}{9a}
        Prove that $(S, \star)$ is a group.
    \end{problem}

    \begin{solution}
    \end{solution}

    \begin{problem}{9b}
        Is $(S, \star)$ abelian?
    \end{problem}

    \begin{solution}
    \end{solution}

    \begin{problem}{10}
    \end{problem}

    \begin{solution}
    \end{solution}
    
\end{document}