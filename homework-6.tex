\documentclass[12pt, letterpaper]{article}
\usepackage{amsfonts}
\usepackage{amssymb}
\usepackage{amsmath}
\usepackage{amsthm}
\usepackage{titling}
\usepackage{mathtools}
\usepackage[shortlabels]{enumitem}
\usepackage[margin=1in]{geometry}

\setlength{\droptitle}{-8ex}
\pretitle{\begin{flushleft}\large}
\posttitle{\par\end{flushleft}}
\preauthor{\begin{flushleft}\large}
\postauthor{\end{flushleft}}
\predate{\begin{flushleft}\large}
\postdate{\end{flushleft}}

\title{Abstract Algebra (MATH-4620) HOMEWORK 6}
\author{Christian Dean}
\date{November 8, 2023}


\newenvironment{problem}
    [1]
    {\noindent \textbf{Problem #1:}}
    {\vspace{3mm}}

\begin{document}

\maketitle

\noindent\hfil\rule{16cm}{0.4pt}\hfil

\begin{problem}{1}
    \begin{enumerate}
        \setcounter{enumi}{6}

        \item $(n\mathbb{Z}, +, \cdot)$ is a ring. It is also commutative as regular multiplication
        of integers is commutative. It has unity only when $n = 1$. But it is not a field, as 
        $n\mathbb{Z}$ is not closed under multiplicative inverses.

        \item $(\mathbb{Z}^+, +, \cdot)$ is not a ring, as it $(\mathbb{Z}^+, +)$ is not an abelian
        group. Namely, $(\mathbb{Z}^+, +)$ contains no additive idenity.

        \item $(\mathbb{Z} \times \mathbb{Z}, +, \cdot)$ is a ring. It is also commutative as 
        regular multiplication of integers is commutative. It also has unity, as $(1, 1)$ is the
        multiplicative identity. But it is not a field, as $\mathbb{Z}$ is not closed under 
        multiplicative inverses.

        \item $(2\mathbb{Z} \times \mathbb{Z}, +, \cdot)$ is a ring. It is also commutative as 
        regular multiplication of integers is commutative. But it does not have unity, as $(1, 1)$ 
        is not in $2\mathbb{Z} \times \mathbb{Z}$. And thus, it also cannot be a field.

        \item Let $S = \{a+b\sqrt{2}\ | a, b \in \mathbb{Z}\}$. $R = (S, +, \cdot)$ is a ring as 
        $(S, +) \cong (\mathbb{Z}^2, +)$, and is thus an abelian group, regular multiplication between 
        real numbers is associative, and the distributive property holds among real numbers. It is also 
        commutative as regular multiplication among real numbers is commutative. It also has unity, as
        $1 \in S$. 


        \bigskip\noindent
        But it is not a field. Fix $a + b\sqrt{2} \in S$, and suppose it has an inverse, $c + d\sqrt{2}
        \in S$. Then $(a + b\sqrt{2})(c + d\sqrt{2}) = 1 \Rightarrow ac + ad\sqrt{2} + bc\sqrt{2} + 2bd 
        = 1 \Rightarrow \sqrt{2}(ad + bc) = 1 - ac - 2bd$. A solution to this equation would be when
        $ad + bc = 0$ and $1 - ac - 2bd = 0$. Thus, keeping $a$ and $b$ fixed, we can solve for $c$
        and $d$ in terms of $a$ and $b$. (The algebra to find these formulas was straightforward but
        tedious, so I opted not to include it for clarity):

        $$
            c = \frac{a}{a^2-2b^2} \;\;\; d = \frac{-b}{a^2-2b^2}
        $$

        Thus there is a formula to compute the multiplicative inverse of an arbitrary element in $R$:
        For any choice of $a + b\sqrt{2} \in R$, the multiplicative inverse $(a + b\sqrt{2})^{-1} = 
        (\frac{a}{a^2-2b^2} - \frac{b}{a^2-2b^2}\sqrt{2})$. But, the formula will not always produce 
        an integer, and thus not every element of $R$ will have an inverse.
        

        \item Let $S = \{a+b\sqrt{2}\ | a, b \in \mathbb{Q}\}$. $R = (S, +, \cdot)$ is a ring as 
        $(S, +) \cong (\mathbb{Q}^2, +)$, and is thus an abelian group, regular multiplication 
        between real numbers is associative, and the distributive property holds among real numbers.
        It is also commutative as regular multiplication among real numbers is commutative. It also 
        has unity, as $1 \in S$.

        \bigskip\noindent
        Further, given the formula found in the previous question, every non-zero element of $R$ will 
        have a multiplicative inverse. Since the rationals are closed under multiplication and addition, the
        only way the above formula would fail would be if the denominators were ever zero. But this
        cannot occur. \emph{Proof}: If the denominators were zero, then $a^2-2b^2 = 0 \Rightarrow
        a^2 = 2b^2 \Rightarrow a = \pm\sqrt{2b^2} = \pm b\sqrt{2}$. Thus, as $a, b \neq 0$, and
        $b \in \mathbb{Q}$, $a$ would need to be irrational. But as $a \in \mathbb{Q}$, this cannot
        occur. Thus the denominators of the formula will never be zero.
        
        \item $(i\mathbb{R}, +, \cdot)$ is a ring, as $(i\mathbb{R}, +)$ forms an abelian group,
        multiplication among imaginary numbers is associative, and the distributive property
        holds among imaginary numbers. It is also commutative as multiplication among imaginary
        numbers commutes. But it does not have unity, as $1 \notin i\mathbb{R}$, and no other
        element of $i\mathbb{R}$ can act as the identity. \emph{Proof}: We will proceed by 
        contradiction. Suppose $i\mathbb{R}$ is a field, and let $ei$ be the identity element,
        where $e \in \mathbb{R}$. Then $(ei)(ei) = ei \Rightarrow -e^2 = e \Rightarrow e^2 = -e$.
        Note $ei \neq 0$, since from observation $0$ cannot be the multiplicative identity. Thus
        $e^2 = -e$ is a contradiction, as any non-zero real number squared is positive. Thus the 
        proof is complete. 
    \end{enumerate}
\end{problem}

\begin{problem}{2}
    \begin{enumerate}
        \setcounter{enumi}{43}

        \item (\textbf{a}) \emph{Proof:} Let $I$ denote the set of idemponet elements in a commutative
        ring. Let $a, b \in I$. Then $ab = (a^2)(b^2) = aabb = abab = (ab)^2 \Rightarrow ab \in I$.

        \bigskip\noindent
        (\textbf{b}) There are 16 idemponets in the ring $\mathbb{Z}_6 \times \mathbb{Z}_{12}$: 
        $\{(0, 0), (0, 1), (0, 4), (0, 9), \\ (3, 0), (3, 1), (3, 4),(3, 9), (1, 0), (1, 1), 
        (1, 4), (1, 9), (4, 0), (4, 1), (4, 4), (4, 9)\}$

        \setcounter{enumi}{45}

        \item Let $N$ denote the set of nilpotent elements in a commutative ring. Let $a, b \in N$.
        Consider $(a + b)^{nm}$. By the binomial theorem, every term in the expansion of $(a + b)$ will
        have the form $\binom{nm}{k} \cdot a^{nm-k} \times b^k$, where $0 \le k \le nm$. We will proceed
        by cases.

        \bigskip\noindent
        \textbf{Case 1:} $k \ge m \Rightarrow b^k = 0 \Rightarrow \binom{nm}{k} \cdot a^{nm-k} \times b^k = 0$.
        \textbf{Case 2:} $k \le m$. It is enough to suppose that $k = m - 1$ and prove that $nm-k = nm - (m-1) \ge n$,
        since if it is true that $nm - (m-1) \ge n$, then $nm - k \ge n$ for any other value of $k \le m$.

        \bigskip\noindent
        Note that $nm - n(m - 1)  = n(m - (m - 1)) = n(m - m + 1) = n(1) = n \Rightarrow nm - n(m - 1) \ge n$. Thus,
        as $n(m - 1) \ge (m - 1)$, then $nm - (m-1) \ge n \Rightarrow a^{nm-k} = 0 \Rightarrow \binom{nm}{k} \cdot a^{nm-k} 
        \times b^k = 0$.

        \bigskip\noindent
        Thus, as every term of the expansion of $(a + b)^{nm}$ is zero, then $(a+b)^{nm} = 0 \Rightarrow (a+b) \in N$, and the
        proof is complete.
    \end{enumerate}
\end{problem}

\begin{problem}{3}
    \begin{enumerate}[(a)]
        \item $\mathbb{Z} \times \mathbb{Z}$ has characteristic $0$, but is not an integral domain
        as $(1, 0)$ and $(0, 1)$ are zero divisors.

        \item $\mathbb{R}$ with regular multiplication and addition is a field with a characteristic
        of $0$, as the additive order of the multiplicative identity $1$ is infinite. And it is not
        isomorphic to $\mathbb{Q}$ as $\mathbb{R}$ is uncountably infinite but $\mathbb{Q}$ is 
        countable.
        
        \item Conider the ring $M_2(\mathbb{Z}_5)$ with coordinate-wise addition and matrix 
        multiplication. It has characteristic $5$, since adding an element of the ring to itself
        $5$ times would be equivelent to multiplying each entry of the matrix by $5$, which would 
        mean each entry would be $0$, mod $5$. Further, it has a finite number of elements: 
        $5^4 = 625$. But it is not a field, as some matrices have zero determinants, and thus have 
        no multiplicative inverses.
        
        \item $\mathbb{Z}_5[x]$
    \end{enumerate}
\end{problem}

\begin{problem}{4}
    \emph{Proof:} We will proceed by proving the contrapositive: If the characteristic of a ring $R$
    is composite, it is not a field.

    \bigskip\noindent
    Let $(R, +, \times)$ be a ring such that $char(R) = n$ is composite. Then, there exists two 
    positive integers $a$ and $b$ such that $a,b \neq 1$, $a, b \neq n$, and $n = ab$. Let 
    $x = \sum_{1}^{a}1_R$ and $y = \sum_{1}^{b}1_R$. Note $x,y \neq 0_R$ as the characteristic is 
    $n$, and $a, b \le n$.

    \bigskip\noindent
    Consider $ab = \sum_{1}^{a}1_R \times \sum_{1}^{b}1_R$. The result of this product will be 
    summing $1_R \times 1_R$ $ab$-times $\Rightarrow \sum_{1}^{a}1_R \times \sum_{1}^{b}1_R = ab 
    \cdot 1_R = n \cdot 1_R = 0_R \Rightarrow ab = 0_R$.

    \bigskip\noindent
    Thus, as $a$ and $b$ are two elements of the ring that are non-zero, yet have a product of
    zero, then $a$ and $b$ are zero-divisors. And as a zero-divisior cannot be a unit, then 
    $R$ has elements that are not units, which demonstrates $R$ is not a field. Thus the proof
    is complete.
\end{problem}

\begin{problem}{5}
    \begin{enumerate}[(a)]
        \item \emph{Proof for $R$}: $(R, +) \cong (\mathbb{Z}^2, +)$, thus $(R, +)$ is an abelian 
        group. Multiplication is associative as regular multiplication of real numbers is associative,
        and the distributive property holds for real numbers. Addtionally, $R$ is commutative: Let
        $a + b\sqrt{3}, c + d\sqrt{3} \in R$. Then $(a + b\sqrt{3})(c + d\sqrt{3}) = ac + ad\sqrt{3}
        + bc\sqrt{3} + 3bd = ca + cb\sqrt{3} + da\sqrt{3} + 3db = (c + d\sqrt{3})(a + b\sqrt{3})$.

        \bigskip\noindent 
        \emph{Proof for $R'$}: $R'$ is a subring of $M_2(\mathbb{Z})$, as $(R', +)$ is a subgroup of
        $(M_2(\mathbb{Z}), +)$, $R'$ contains the multiplicative identity, when $a = 1, b = 0$, and 
        it is closed under matrix multiplication: Let $A = \begin{pmatrix*}[r] a & 3b \\ b & a 
        \end{pmatrix*}, B = \begin{pmatrix*}[r] c & 3d \\ d & c \end{pmatrix*} \in R'$. Then $AB = 
        C = \begin{pmatrix*}[r] ac + 3bd & 3(ad + bc) \\ ad + bc & ac + 3bd\end{pmatrix*}$, and note
        $c_{1,1} = c_{2,2}$, $c_{1,2} = 3c_{2,1}$, and $ad + bc, ac + 3bd \in \mathbb{Z}$. Further $R'$ 
        is commutative, as $BA = \begin{pmatrix*}[r] ca + 3db & 3(cb + da) \\ da + cb & ca + 3db
        \end{pmatrix*} = \begin{pmatrix*}[r] ac + 3bd & 3(ad + bc) \\ ad + bc & ac + 3bd\end{pmatrix*}
        \\= AB$, as regular addition and multiplication of the integers is commutative.

        \item 

    \end{enumerate}
\end{problem}

\end{document}